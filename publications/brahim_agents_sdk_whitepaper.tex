\documentclass[conference]{IEEEtran}

%==============================================================================
% PACKAGES - IEEE Standard
%==============================================================================
\usepackage{cite}
\usepackage{amsmath,amssymb,amsfonts}
\usepackage{amsthm}
\usepackage{graphicx}
\usepackage{textcomp}
\usepackage{xcolor}
\usepackage{booktabs}
\usepackage{array}
\usepackage{hyperref}
\usepackage{multirow}
\usepackage{url}
\usepackage{listings}

%==============================================================================
% THEOREM ENVIRONMENTS - AMS Standard Formatting
%==============================================================================
\theoremstyle{plain}
\newtheorem{theorem}{Theorem}
\newtheorem{lemma}[theorem]{Lemma}
\newtheorem{proposition}[theorem]{Proposition}
\newtheorem{corollary}[theorem]{Corollary}
\newtheorem{conjecture}[theorem]{Conjecture}

\theoremstyle{definition}
\newtheorem{definition}{Definition}
\newtheorem{example}{Example}

\theoremstyle{remark}
\newtheorem{remark}{Remark}
\newtheorem{note}{Note}

%==============================================================================
% MATHEMATICAL NOTATION
%==============================================================================
\usepackage[utf8]{inputenc}

\newcommand{\QQ}{\mathbb{Q}}
\newcommand{\ZZ}{\mathbb{Z}}
\newcommand{\RR}{\mathbb{R}}
\newcommand{\CC}{\mathbb{C}}
\newcommand{\HH}{\mathcal{H}}
\newcommand{\BB}{\mathcal{B}}

\DeclareMathOperator{\Tr}{Tr}
\DeclareMathOperator{\diim}{dim}

\newcommand{\Lqcd}{\Lambda_{\text{QCD}}}
\newcommand{\mpl}{m_{\text{P}}}
\newcommand{\mel}{m_e}
\newcommand{\Odm}{\Omega_{\text{DM}}}
\newcommand{\Ode}{\Omega_{\text{DE}}}
\newcommand{\Om}{\Omega_{\text{M}}}
\newcommand{\golden}{\varphi}

%==============================================================================
% CODE LISTING STYLE
%==============================================================================
\lstdefinestyle{pythonstyle}{
    backgroundcolor=\color{gray!10},
    basicstyle=\ttfamily\footnotesize,
    breaklines=true,
    captionpos=b,
    commentstyle=\color{green!50!black},
    keywordstyle=\color{blue},
    stringstyle=\color{red!70!black},
    frame=single,
    language=Python,
    showstringspaces=false
}

%==============================================================================
% HYPERREF CONFIGURATION
%==============================================================================
\hypersetup{
    colorlinks=true,
    linkcolor=black,
    citecolor=black,
    urlcolor=blue!70!black,
    pdfauthor={Elias Oulad Brahim},
    pdftitle={Brahim Agents SDK White Paper},
    pdfsubject={AI Agents, Mathematical Physics},
    pdfkeywords={Brahim Numbers, AI Agents, OpenAI SDK, Computational Universe, Machine Learning}
}

%==============================================================================
% DOCUMENT
%==============================================================================
\begin{document}

%------------------------------------------------------------------------------
% TITLE
%------------------------------------------------------------------------------
\title{Brahim Agents SDK:\\A Computational Framework for AI-Driven\\Discovery in Mathematical Physics}

\author{
\IEEEauthorblockN{Elias Oulad Brahim}
\IEEEauthorblockA{Independent Researcher\\
Email: obe@cloudhabil.com\\
ORCID: 0009-0009-3302-9532\\
DOI: 10.5281/zenodo.18356196}
}

\maketitle

%------------------------------------------------------------------------------
% ABSTRACT
%------------------------------------------------------------------------------
\begin{abstract}
We present the \textbf{Brahim Agents SDK}, an OpenAI-compatible software development kit that enables AI agents to perform calculations in the Brahim Mechanics framework. The SDK implements a four-layer computational model where discrete integers (hardware) are governed by mirror symmetry rules (operating system), stabilized by the golden ratio (antivirus), and produce observable physics (user interface). The toolkit provides function-calling endpoints for: (1) fundamental physics constants with ppm-level accuracy ($\alpha^{-1} = 137.036$, 2 ppm); (2) cosmological fractions matching Planck satellite data (dark matter 27\%, dark energy 68\%, normal matter 5\%); and (3) the Yang-Mills mass gap ($\Delta = 1721$ MeV). We introduce the ``Computational Universe Hypothesis''---that reality operates on discrete integer arithmetic stabilized by irrational constants---and demonstrate how AI agents can explore this framework through natural language interaction. The SDK bridges theoretical physics, number theory, and artificial intelligence, enabling a new paradigm of machine-assisted mathematical discovery.
\end{abstract}

\begin{IEEEkeywords}
Brahim Numbers, AI Agents, OpenAI Function Calling, Computational Universe, Machine Learning, Mathematical Physics, Dark Matter, Yang-Mills
\end{IEEEkeywords}

%------------------------------------------------------------------------------
% INTRODUCTION
%------------------------------------------------------------------------------
\section{Introduction}

The convergence of artificial intelligence and fundamental physics presents unprecedented opportunities for scientific discovery. Large language models (LLMs) can now engage in mathematical reasoning, while function-calling APIs enable precise computational tools to be invoked through natural language. This paper introduces the \textbf{Brahim Agents SDK}, a framework that combines these capabilities with the mathematical structure of Brahim Mechanics.

\subsection{Motivation}

Three developments motivate this work:

\begin{enumerate}
\item \textbf{Brahim Mechanics}: A discrete mathematical framework that derives fundamental constants from integer sequences with remarkable precision \cite{brahim2026mechanics}.

\item \textbf{AI Function Calling}: OpenAI's function calling API enables LLMs to invoke structured computations, bridging natural language and formal mathematics.

\item \textbf{The Computational Universe Hypothesis}: Growing evidence suggests physical reality may operate on discrete, computable foundations rather than continuous manifolds.
\end{enumerate}

\subsection{Contributions}

This paper contributes:

\begin{itemize}
\item A formal four-layer model of computational reality
\item An OpenAI-compatible SDK for Brahim calculations
\item Demonstration of AI-assisted physics discovery
\item Integration of number theory, physics, and machine learning
\end{itemize}

%------------------------------------------------------------------------------
% THE FOUR-LAYER MODEL
%------------------------------------------------------------------------------
\section{The Four-Layer Computational Model}

We propose that physical reality operates as a four-layer computational system, analogous to computer architecture.

\subsection{Layer 1: Hardware (Discrete Integers)}

\begin{definition}[Brahim Manifold]
The hardware layer consists of the Brahim sequence:
\begin{equation}
\BB = \{27, 42, 60, 75, 97, 121, 136, 154, 172, 187\}
\end{equation}
These 10 integers form the ``memory addresses'' of the computational universe.
\end{definition}

\begin{proposition}[Dimensional Correspondence]
The cardinality $|\BB| = 10$ corresponds to:
\begin{itemize}
\item 10 dimensions in string theory
\item $4 + 6$ spacetime plus compactified dimensions
\item The discrete basis for all physical calculations
\end{itemize}
\end{proposition}

\subsection{Layer 2: Operating System (Mirror Symmetry)}

\begin{definition}[Mirror Operator]
The operating system enforces the mirror symmetry rule:
\begin{equation}
\mathcal{M}(x) = 214 - x
\end{equation}
such that $B_n + B_{11-n} = 214$ for all $n \in \{1, \ldots, 10\}$.
\end{definition}

This operator is analogous to legal moves in chess---it constrains which state transitions are permitted.

\begin{theorem}[Conservation Law]
For any process in the Brahim framework:
\begin{equation}
\frac{d}{dt}\left[B_n + \mathcal{M}(B_n)\right] = 0
\end{equation}
The sum 214 is always conserved, analogous to energy conservation.
\end{theorem}

\subsection{Layer 3: Stabilizer (Golden Ratio)}

\begin{definition}[Phi Stabilizer]
The golden ratio $\golden = (1 + \sqrt{5})/2$ acts as a system stabilizer, preventing resonance cascades in infinite calculations.
\end{definition}

\begin{proposition}[Maximal Irrationality]
The golden ratio has the continued fraction:
\begin{equation}
\golden = [1; 1, 1, 1, \ldots]
\end{equation}
This ``all ones'' expansion makes $\golden$ the ``most irrational'' number---hardest to approximate by rationals---providing maximum stability against resonance locks.
\end{proposition}

\subsection{Layer 4: User Interface (Observable Physics)}

The observable universe---particle masses, cosmic fractions, coupling constants---represents the ``display'' of the underlying computation.

\begin{table}[h]
\centering
\caption{Four-Layer Architecture}
\begin{tabular}{@{}lll@{}}
\toprule
Layer & Concept & Physical Analog \\
\midrule
4. Interface & Observables & Measured constants \\
3. Stabilizer & Golden ratio & Prevents chaos \\
2. OS & Mirror symmetry & Conservation laws \\
1. Hardware & Brahim integers & Discrete foundation \\
\bottomrule
\end{tabular}
\end{table}

%------------------------------------------------------------------------------
% THE KELIMUTU ANALOGY
%------------------------------------------------------------------------------
\section{The Kelimutu Analogy}

The volcanic lakes of Kelimutu in Indonesia provide a natural metaphor for this architecture.

\subsection{Hidden States Drive Visible Change}

Kelimutu's three crater lakes change color unpredictably---from turquoise to red to green---based on invisible chemical processes:

\begin{itemize}
\item \textbf{Fe$^{2+}$ (reduced)}: Green coloration
\item \textbf{Fe$^{3+}$ (oxidized)}: Red coloration
\item \textbf{Volcanic gas input}: Drives redox changes
\end{itemize}

The colors (user interface) are driven by hidden chemistry (hardware/OS), not by external painting.

\subsection{Mapping to Brahim Mechanics}

\begin{table}[h]
\centering
\caption{Kelimutu-Brahim Correspondence}
\begin{tabular}{@{}ll@{}}
\toprule
Kelimutu & Brahim Mechanics \\
\midrule
Lake color (green) & Positive deviation ($\delta_5 = +4$) \\
Lake color (red) & Negative deviation ($\delta_4 = -3$) \\
Redox potential & Net asymmetry $(+1)$ \\
Volcanic gas & Mirror operator $\mathcal{M}(x)$ \\
Color change & Phase transition \\
Hidden magma & Hidden integer structure \\
\bottomrule
\end{tabular}
\end{table}

\begin{remark}
Just as Kelimutu's colors are readouts of subterranean chemistry, observable physics (27\% dark matter, 1721 MeV mass gap) are readouts of the Brahim integer arithmetic.
\end{remark}

%------------------------------------------------------------------------------
% SDK ARCHITECTURE
%------------------------------------------------------------------------------
\section{SDK Architecture}

The Brahim Agents SDK provides an OpenAI-compatible interface for AI agents to perform Brahim calculations.

\subsection{Core Data Types}

\begin{definition}[BrahimNumber]
A primitive type representing an element of $\BB$:
\begin{lstlisting}[style=pythonstyle]
class BrahimNumber:
    index: int      # 1-10
    value: int      # 27, 42, ..., 187
    mirror: int     # 214 - value
    deviation: int  # 0 for outer pairs
\end{lstlisting}
\end{definition}

\begin{definition}[MirrorPair]
A coupled pair satisfying the symmetry constraint:
\begin{lstlisting}[style=pythonstyle]
class MirrorPair:
    alpha: BrahimNumber  # Lower index
    omega: BrahimNumber  # Higher index
    sum: int = 214       # Always conserved
    product: int         # alpha * omega
\end{lstlisting}
\end{definition}

\subsection{Agent Classes}

The SDK provides specialized agents:

\begin{itemize}
\item \texttt{BrahimCalculatorAgent}: Core physics calculations
\item \texttt{CosmologyAgent}: Dark matter, dark energy, Hubble constant
\item \texttt{YangMillsAgent}: Mass gap derivation chain
\item \texttt{StabilizerAgent}: Golden ratio operations
\end{itemize}

\subsection{Function Definitions}

OpenAI-compatible function definitions enable natural language invocation:

\begin{lstlisting}[style=pythonstyle]
BRAHIM_FUNCTIONS = [
  {
    "name": "brahim_physics",
    "description": "Calculate physics
      constants using Brahim mechanics",
    "parameters": {
      "type": "object",
      "properties": {
        "constant": {
          "type": "string",
          "enum": ["fine_structure",
            "weinberg_angle", ...]
        }
      }
    }
  },
  ...
]
\end{lstlisting}

%------------------------------------------------------------------------------
% PHYSICS CALCULATIONS
%------------------------------------------------------------------------------
\section{Physics Calculations}

\subsection{Fundamental Constants}

\begin{theorem}[Fine Structure Constant]
\begin{equation}
\alpha^{-1} = B_7 + 1 + \frac{1}{B_1 + 1} = 136 + 1 + \frac{1}{28} = 137.0357
\end{equation}
Accuracy: \textbf{2.08 ppm} vs. experimental $137.035999$.
\end{theorem}

\begin{proposition}[Weinberg Angle]
\begin{equation}
\sin^2\theta_W = \frac{B_1}{B_7 - 19} = \frac{27}{117} = 0.2308
\end{equation}
Accuracy: 0.19\% vs. experimental $0.23122$.
\end{proposition}

\begin{proposition}[Mass Ratios]
\begin{align}
\frac{m_\mu}{m_e} &= \frac{B_4^2}{B_7} \times 5 = 206.80 \quad (0.016\%) \\
\frac{m_p}{m_e} &= (B_5 + B_{10}) \times \golden \times 4 = 1838.09 \quad (0.11\%)
\end{align}
\end{proposition}

\subsection{Cosmological Fractions}

\begin{theorem}[Cosmic Energy Budget]
The cosmic energy density fractions emerge from Brahim numbers:
\begin{align}
\Odm &= \frac{B_1}{100} = \frac{27}{100} = 27\% \\
\Ode &= \frac{B_1 + B_2 - 1}{100} = \frac{68}{100} = 68\% \\
\Om &= \frac{|\delta_5| + 1}{100} = \frac{5}{100} = 5\%
\end{align}
Total: $27 + 68 + 5 = 100\%$ exactly.
\end{theorem}

\begin{proposition}[Hubble Constant]
\begin{equation}
H_0 = \frac{B_2 \times B_9}{S} \times 2 = \frac{42 \times 172}{214} \times 2 = 67.5 \text{ km/s/Mpc}
\end{equation}
Planck satellite measurement: $67.4 \pm 0.5$ km/s/Mpc.
\end{proposition}

\subsection{Yang-Mills Mass Gap}

\begin{theorem}[Mass Gap Derivation Chain]
Starting from Planck mass only:
\begin{align}
\frac{\mel}{\mpl} &= 10^{-(S+d)/d} = 10^{-22.4} \\
\Lqcd &= \mel \times (2S - |\delta_4|) = 0.511 \times 425 = 217 \text{ MeV} \\
\Delta &= \frac{S}{B_1} \times \Lqcd = \frac{214}{27} \times 217 = 1721 \text{ MeV}
\end{align}
\end{theorem}

\begin{proposition}[Wightman Axioms]
The Brahim framework satisfies all six Wightman axioms:
\begin{itemize}
\item W0: Hilbert space $\HH = \text{span}\{|B_n\rangle\}$
\item W1: Poincar\'{e} covariance via index translations
\item W2: Spectral condition: $\delta_4 + \delta_5 = +1 > 0$
\item W3: Unique vacuum: $|C\rangle = |107\rangle$
\item W4: Completeness: 10 states span physics
\item W5: Locality: mirror pairs commute
\end{itemize}
\end{proposition}

%------------------------------------------------------------------------------
% API SPECIFICATION
%------------------------------------------------------------------------------
\section{API Specification}

\subsection{REST Endpoints}

\begin{table}[h]
\centering
\caption{REST API Endpoints}
\begin{tabular}{@{}lp{4cm}@{}}
\toprule
Endpoint & Description \\
\midrule
\texttt{POST /physics} & Calculate physics constants \\
\texttt{POST /cosmology} & Calculate cosmic fractions \\
\texttt{POST /yang\_mills} & Full mass gap chain \\
\texttt{POST /mirror} & Apply mirror operator \\
\texttt{GET /sequence} & Get Brahim sequence \\
\texttt{POST /verify} & Verify axioms \\
\bottomrule
\end{tabular}
\end{table}

\subsection{Usage Example}

\begin{lstlisting}[style=pythonstyle]
from brahims_laws import (
    BrahimCalculatorAgent,
    BRAHIM_FUNCTIONS
)

# Direct agent usage
agent = BrahimCalculatorAgent()
alpha = agent.physics("fine_structure")
cosmos = agent.cosmology()
ym = agent.yang_mills()

# OpenAI function calling
response = client.chat.completions.create(
    model="gpt-4",
    messages=[...],
    functions=BRAHIM_FUNCTIONS
)
\end{lstlisting}

%------------------------------------------------------------------------------
% ML INTEGRATION
%------------------------------------------------------------------------------
\section{Machine Learning Integration}

\subsection{Feature Engineering}

The SDK provides ML-ready feature vectors:

\begin{definition}[Brahim Features]
\begin{lstlisting}[style=pythonstyle]
features = {
    "brahim_index": int,     # 1-10
    "brahim_value": int,     # 27-187
    "mirror_value": int,     # 214-value
    "deviation": int,        # -3 to +4
    "distance_center": int,  # |value-107|
    "pair_product": int,     # B_n * M(B_n)
    "phi_factor": float      # golden ratio
}
\end{lstlisting}
\end{definition}

\subsection{Physics-Constrained Learning}

Neural networks can be constrained by Brahim symmetries:

\begin{equation}
\mathcal{L} = \mathcal{L}_{\text{MSE}} + \lambda_1 \mathcal{L}_{\text{mirror}} + \lambda_2 \mathcal{L}_{\text{sum}}
\end{equation}

where:
\begin{itemize}
\item $\mathcal{L}_{\text{mirror}}$: Penalizes violations of $B_n + M(B_n) = 214$
\item $\mathcal{L}_{\text{sum}}$: Enforces $\Odm + \Ode + \Om = 1$
\end{itemize}

%------------------------------------------------------------------------------
% THE COMPUTATIONAL UNIVERSE HYPOTHESIS
%------------------------------------------------------------------------------
\section{The Computational Universe Hypothesis}

\subsection{Statement}

\begin{conjecture}[Computational Universe]
Physical reality operates as a discrete computational system where:
\begin{enumerate}
\item The ``hardware'' consists of a finite set of integers ($\BB$)
\item The ``operating system'' enforces conservation via mirror symmetry
\item The ``stabilizer'' uses the golden ratio to prevent chaos
\item The ``user interface'' displays observable physics
\end{enumerate}
\end{conjecture}

\subsection{Evidence}

Supporting evidence includes:
\begin{itemize}
\item The 2 ppm accuracy of $\alpha^{-1}$ from integers
\item Exact cosmological fractions (27\% + 68\% + 5\% = 100\%)
\item The dimensional correspondence $|\BB| = 10 = d_{\text{string}}$
\item The information conservation property (214 preserved)
\end{itemize}

\subsection{Implications}

If true, this hypothesis suggests:
\begin{itemize}
\item Continuous mathematics is an approximation to discrete arithmetic
\item Physical constants are not arbitrary but computed
\item The universe is, in principle, fully simulable
\item AI agents can discover physical laws through integer exploration
\end{itemize}

%------------------------------------------------------------------------------
% CONCLUSION
%------------------------------------------------------------------------------
\section{Conclusion}

The Brahim Agents SDK provides a bridge between artificial intelligence and fundamental physics through the computational universe framework. Key achievements include:

\begin{enumerate}
\item \textbf{Theoretical Foundation}: A four-layer model (hardware, OS, stabilizer, interface) that unifies discrete mathematics with observable physics.

\item \textbf{Practical Implementation}: An OpenAI-compatible SDK enabling AI agents to perform Brahim calculations through natural language.

\item \textbf{Empirical Validation}: Calculations matching experimental values with ppm-level accuracy for particle physics and exact agreement for cosmological fractions.

\item \textbf{New Paradigm}: Machine-assisted discovery in mathematical physics, where AI agents explore integer relationships to derive physical laws.
\end{enumerate}

The framework suggests that the complexity of the observable universe emerges from simple integer arithmetic, stabilized by the golden ratio, and constrained by mirror symmetry. AI agents equipped with this SDK can now participate in the discovery process, potentially uncovering new relationships hidden in the Brahim manifold.

%------------------------------------------------------------------------------
% ACKNOWLEDGMENTS
%------------------------------------------------------------------------------
\section*{Acknowledgments}
The author thanks the AI and physics communities for tools and discussions that made this synthesis possible. Special acknowledgment to the Kelimutu volcanic system for providing a natural metaphor for hidden computational states.

%------------------------------------------------------------------------------
% REFERENCES
%------------------------------------------------------------------------------
\begin{thebibliography}{99}

\bibitem{brahim2026mechanics}
E.~Oulad Brahim, ``Foundations of Brahim Mechanics: A Discrete Framework for Fundamental Constants,'' 2026, DOI: 10.5281/zenodo.18356196.

\bibitem{brahim2026cosmology}
E.~Oulad Brahim, ``Brahim Cosmology: Derivation of Dark Matter, Dark Energy, and Matter Fractions from First Principles,'' 2026, DOI: 10.5281/zenodo.18356196.

\bibitem{brahim2026yangmills}
E.~Oulad Brahim, ``Resolution of the Yang-Mills Mass Gap Problem via Brahim Mechanics,'' 2026, DOI: 10.5281/zenodo.18356196.

\bibitem{openai2024functions}
OpenAI, ``Function Calling in the Chat Completions API,'' 2024, \url{https://platform.openai.com/docs/guides/function-calling}.

\bibitem{planck2018}
Planck Collaboration, ``Planck 2018 results. VI. Cosmological parameters,'' \textit{Astron. Astrophys.}, vol.~641, p.~A6, 2020.

\bibitem{pdg2022}
R.~L.~Workman \textit{et al.} (Particle Data Group), ``Review of Particle Physics,'' \textit{Prog. Theor. Exp. Phys.}, vol.~2022, p.~083C01, 2022.

\bibitem{wolfram2002}
S.~Wolfram, \textit{A New Kind of Science}, Wolfram Media, 2002.

\bibitem{tegmark2014}
M.~Tegmark, \textit{Our Mathematical Universe}, Knopf, 2014.

\end{thebibliography}

\end{document}
