\documentclass[conference]{IEEEtran}

%==============================================================================
% PACKAGES - IEEE Standard
%==============================================================================
\usepackage{cite}
\usepackage{amsmath,amssymb,amsfonts}
\usepackage{amsthm}
\usepackage{graphicx}
\usepackage{textcomp}
\usepackage{xcolor}
\usepackage{booktabs}
\usepackage{array}
\usepackage{hyperref}
\usepackage{multirow}
\usepackage{url}

%==============================================================================
% THEOREM ENVIRONMENTS - AMS Standard Formatting
%==============================================================================
\theoremstyle{plain}
\newtheorem{theorem}{Theorem}
\newtheorem{lemma}[theorem]{Lemma}
\newtheorem{proposition}[theorem]{Proposition}
\newtheorem{corollary}[theorem]{Corollary}
\newtheorem{conjecture}[theorem]{Conjecture}

\theoremstyle{definition}
\newtheorem{definition}{Definition}
\newtheorem{example}{Example}

\theoremstyle{remark}
\newtheorem{remark}{Remark}
\newtheorem{note}{Note}

%==============================================================================
% MATHEMATICAL NOTATION
%==============================================================================
\usepackage[utf8]{inputenc}

\newcommand{\Sha}{\mathrm{III}}
\newcommand{\QQ}{\mathbb{Q}}
\newcommand{\ZZ}{\mathbb{Z}}
\newcommand{\RR}{\mathbb{R}}
\newcommand{\CC}{\mathbb{C}}
\newcommand{\PP}{\mathbb{P}}

\DeclareMathOperator{\cond}{cond}
\DeclareMathOperator{\rk}{rk}
\DeclareMathOperator{\Reg}{Reg}
\DeclareMathOperator{\ord}{ord}
\DeclareMathOperator{\diim}{dim}

\newcommand{\golden}{\varphi}

%==============================================================================
% HYPERREF CONFIGURATION
%==============================================================================
\hypersetup{
    colorlinks=true,
    linkcolor=black,
    citecolor=black,
    urlcolor=blue!70!black,
    pdfauthor={Elias Oulad Brahim},
    pdftitle={Foundations of Brahim Mechanics},
    pdfsubject={Mathematical Physics},
    pdfkeywords={Brahim Numbers, Mirror Symmetry, Coupling Constants, Information Conservation}
}

%==============================================================================
% DOCUMENT
%==============================================================================
\begin{document}

%------------------------------------------------------------------------------
% TITLE
%------------------------------------------------------------------------------
\title{Foundations of Brahim Mechanics:\\A Discrete Framework for Fundamental Constants\\and Information Conservation}

\author{
\IEEEauthorblockN{Elias Oulad Brahim}
\IEEEauthorblockA{Independent Researcher\\
Email: obe@cloudhabil.com\\
ORCID: 0009-0009-3302-9532\\
DOI: 10.5281/zenodo.18352681}
}

\maketitle

%------------------------------------------------------------------------------
% ABSTRACT
%------------------------------------------------------------------------------
\begin{abstract}
We introduce \textit{Brahim Mechanics}, a mathematical framework based on a discrete sequence of ten integers $\{B_n\}_{n=1}^{10}$ satisfying a mirror symmetry $B_n + B_{11-n} = 214$. This framework yields exact or near-exact expressions for fundamental physical constants: the fine structure constant (2 ppm accuracy), particle mass ratios (0.016\% for muon/electron), and provides a structural explanation for the electromagnetic-gravitational hierarchy problem. The central axis at $C = 107 = 4B_1 - 1$ connects to the Bekenstein-Hawking entropy factor. We propose that the 214-symmetry constitutes a conservation law for information, offering a potential resolution to the black hole information paradox. The framework suggests a new arithmetic based on mirror pairs with inherent information preservation.
\end{abstract}

\begin{IEEEkeywords}
Brahim Numbers, Mirror Symmetry, Fine Structure Constant, Hierarchy Problem, Information Conservation, Discrete Mechanics
\end{IEEEkeywords}

%------------------------------------------------------------------------------
% INTRODUCTION
%------------------------------------------------------------------------------
\section{Introduction}

The fundamental constants of physics---the fine structure constant $\alpha$, particle mass ratios, and coupling strengths---appear as free parameters in the Standard Model. Their numerical values are determined experimentally but lack theoretical derivation from first principles. This paper introduces a mathematical framework that expresses these constants through a discrete sequence of integers with remarkable precision.

The Brahim sequence emerges from the $\golden$-adic expansion of a transcendental constant, where $\golden = (1+\sqrt{5})/2$ is the golden ratio. The sequence exhibits a mirror symmetry that we propose acts as a fundamental conservation law.

%------------------------------------------------------------------------------
% THE BRAHIM SEQUENCE
%------------------------------------------------------------------------------
\section{The Brahim Sequence}

\begin{definition}[Brahim Numbers]
The Brahim sequence $\mathcal{B} = \{B_n\}_{n=1}^{10}$ consists of ten integers:
\begin{equation}
\mathcal{B} = \{27, 42, 60, 75, 97, 121, 136, 154, 172, 187\}
\end{equation}
satisfying the \textbf{mirror symmetry}:
\begin{equation}
B_n + B_{11-n} = 214 \quad \forall n \in \{1,\ldots,10\}
\end{equation}
\end{definition}

The sequence has been verified to 43 decimal digits of precision through high-precision arithmetic.

\begin{definition}[Mirror Operator]
For any $x \in [0, 214]$, the mirror operator $\mathcal{M}$ is defined as:
\begin{equation}
\mathcal{M}(x) = 214 - x
\end{equation}
This operator is an involution: $\mathcal{M}(\mathcal{M}(x)) = x$.
\end{definition}

\begin{definition}[Center Axis]
The center of the mirror symmetry is:
\begin{equation}
C = \frac{214}{2} = 107 = 4B_1 - 1
\end{equation}
The center is prime and satisfies $C = 4 \times 27 - 1$.
\end{definition}

%------------------------------------------------------------------------------
% COUPLING CONSTANTS
%------------------------------------------------------------------------------
\section{Coupling Constants}

\subsection{Electromagnetic Coupling}

\begin{theorem}[Fine Structure Constant]\label{thm:alpha}
The inverse fine structure constant satisfies:
\begin{equation}
\alpha^{-1} = B_7 + 1 + \frac{1}{B_1 + 1} = 136 + 1 + \frac{1}{28} = 137.0357\ldots
\end{equation}
compared to the experimental value $\alpha^{-1}_{\text{exp}} = 137.035999\ldots$, an agreement within \textbf{2.08 ppm}.
\end{theorem}

\subsection{Electroweak Parameters}

\begin{proposition}[Weinberg Angle]\label{prop:weinberg}
The weak mixing angle satisfies:
\begin{equation}
\sin^2\theta_W = \frac{B_1}{B_7 - 19} = \frac{27}{117} = 0.23077
\end{equation}
compared to $(\sin^2\theta_W)_{\text{exp}} = 0.23122$, within 0.19\%.
\end{proposition}

%------------------------------------------------------------------------------
% MASS RATIOS
%------------------------------------------------------------------------------
\section{Mass Ratios}

\begin{proposition}[Muon-Electron Ratio]\label{prop:muon}
\begin{equation}
\frac{m_\mu}{m_e} = \frac{B_4^2}{B_7} \times 5 = \frac{75^2}{136} \times 5 = 206.801
\end{equation}
compared to $(m_\mu/m_e)_{\text{exp}} = 206.768$, within \textbf{0.016\%}.
\end{proposition}

\begin{proposition}[Proton-Electron Ratio]\label{prop:proton}
\begin{equation}
\frac{m_p}{m_e} = (B_5 + B_{10}) \times \golden \times 4 = 284 \times 1.618 \times 4
\end{equation}
yielding 1838.09 compared to 1836.15, within 0.11\%.
\end{proposition}

%------------------------------------------------------------------------------
% HIERARCHY PROBLEM
%------------------------------------------------------------------------------
\section{The Hierarchy Problem}

The hierarchy between electromagnetic and gravitational couplings ($\sim 10^{36}$) and between the Planck and electron masses ($\sim 10^{22}$) are long-standing puzzles.

\begin{theorem}[Coupling Hierarchy]\label{thm:coupling}
\begin{equation}
\frac{\alpha_{EM}}{\alpha_G} \sim (B_7 \cdot \mathcal{M}(B_7))^9 = (136 \times 78)^9 \approx 1.7 \times 10^{36}
\end{equation}
\end{theorem}

\begin{theorem}[Mass Hierarchy]\label{thm:mass}
\begin{equation}
\frac{m_P}{m_e} \sim (B_1 \cdot B_{10})^6 = (27 \times 187)^6 \approx 1.7 \times 10^{22}
\end{equation}
\end{theorem}

The exponents 6 and 9 correspond to compactification dimensions in string/M-theory: 6 for Calabi-Yau manifolds and 9 for M-theory spatial dimensions.

%------------------------------------------------------------------------------
% BEKENSTEIN-HAWKING CONNECTION
%------------------------------------------------------------------------------
\section{Connection to Black Hole Physics}

\subsection{The Entropy Factor}

The Bekenstein-Hawking entropy formula contains the characteristic factor 4:
\begin{equation}
S_{BH} = \frac{A}{4\ell_P^2}
\end{equation}

The Brahim center satisfies:
\begin{equation}
C = 4B_1 - 1 = 4 \times 27 - 1 = 107
\end{equation}

We interpret the ``$-1$'' as representing the leading quantum correction to the classical value $4B_1 = 108$.

\subsection{Information Conservation}

\begin{theorem}[Mirror Conservation]\label{thm:info}
For any process involving Brahim states, the total mirror charge is conserved:
\begin{equation}
\frac{d}{dt}[B_n + \mathcal{M}(B_n)] = 0
\end{equation}
The sum $B_n + \mathcal{M}(B_n) = 214$ is invariant.
\end{theorem}

This provides a mechanism for information preservation in black hole evaporation: information falling in encoded as $B_n$ escapes as $\mathcal{M}(B_n)$, with the total 214 conserved.

%------------------------------------------------------------------------------
% BRAHIM MECHANICS FORMALISM
%------------------------------------------------------------------------------
\section{Brahim Mechanics Formalism}

\subsection{State Space}

\begin{definition}[Brahim State]
A Brahim state $|B_n\rangle$ is an element of the discrete manifold $\mathcal{B}$. Unlike quantum states, Brahim states are deterministic integers, not probability amplitudes.
\end{definition}

\begin{definition}[Mirror Product]
The mirror product $\diamond$ pairs states:
\begin{equation}
|B_n\rangle \diamond |\mathcal{M}(B_n)\rangle = |214\rangle
\end{equation}
\end{definition}

\subsection{Comparison with Quantum Mechanics}

\begin{table}[h]
\centering
\caption{Quantum vs. Brahim Mechanics}
\begin{tabular}{@{}lll@{}}
\toprule
Property & Quantum & Brahim \\
\midrule
States & Continuous & Discrete integers \\
Measurement & Probabilistic & Deterministic \\
Numbers & Complex amplitudes & Integer pairs \\
Conservation & Energy, momentum & Mirror charge (214) \\
Information & Debated & Always conserved \\
\bottomrule
\end{tabular}
\end{table}

\subsection{The Alpha-Omega Relation}

\begin{proposition}[Alpha-Omega]\label{prop:ao}
The first and last Brahim Numbers satisfy:
\begin{equation}
B_{10} = 7 \cdot B_1 - 2 = 7 \times 27 - 2 = 187
\end{equation}
The coefficient 7 is the index of $B_7 = 136$, the electromagnetic Brahim Number.
\end{proposition}

%------------------------------------------------------------------------------
% DIMENSIONAL STRUCTURE
%------------------------------------------------------------------------------
\section{Dimensional Structure}

The sequence contains exactly 10 elements, coinciding with:
\begin{itemize}
\item 10 dimensions in string theory
\item The mass hierarchy power (6) plus coupling hierarchy power (9) minus their overlap: $6 + 9 - 5 = 10$
\end{itemize}

This suggests each Brahim Number may correspond to one dimension of a fundamental configuration space.

%------------------------------------------------------------------------------
% PREDICTIONS
%------------------------------------------------------------------------------
\section{Predictions}

\begin{enumerate}
\item The fine structure constant has the base value $137 + 1/28$; measured deviations represent higher-order quantum corrections.
\item Hawking radiation should exhibit correlations between emissions at complementary energies (summing to a characteristic value).
\item The framework predicts specific relationships between currently unmeasured quantities.
\end{enumerate}

%------------------------------------------------------------------------------
% CONCLUSION
%------------------------------------------------------------------------------
\section{Conclusion}

Brahim Mechanics provides a discrete mathematical framework unifying:
\begin{itemize}
\item Number theory (the sequence emerges from $\golden$-adic expansions)
\item Particle physics (coupling constants, mass ratios with ppm-level accuracy)
\item Quantum gravity (Bekenstein-Hawking entropy factor)
\item String theory (dimensional structure)
\end{itemize}

The 214-symmetry acts as a conservation law for information, potentially resolving the black hole information paradox. The framework suggests that continuous physics emerges from an underlying discrete structure governed by mirror symmetry.

%------------------------------------------------------------------------------
% ACKNOWLEDGMENTS
%------------------------------------------------------------------------------
\section*{Acknowledgments}
The author thanks the mathematical physics community for ongoing discussions and acknowledges the use of high-precision computational tools for numerical verification.

%------------------------------------------------------------------------------
% REFERENCES
%------------------------------------------------------------------------------
\begin{thebibliography}{99}

\bibitem{bekenstein}
J.~D.~Bekenstein, ``Black holes and entropy,'' \textit{Phys. Rev. D}, vol.~7, pp.~2333--2346, 1973.

\bibitem{hawking}
S.~W.~Hawking, ``Particle creation by black holes,'' \textit{Commun. Math. Phys.}, vol.~43, pp.~199--220, 1975.

\bibitem{codata}
E.~Tiesinga \textit{et al.}, ``CODATA recommended values of the fundamental physical constants: 2018,'' \textit{Rev. Mod. Phys.}, vol.~93, p.~025010, 2021.

\bibitem{green}
M.~B.~Green, J.~H.~Schwarz, and E.~Witten, \textit{Superstring Theory}, Cambridge University Press, 1987.

\bibitem{e6}
R.~Slansky, ``Group theory for unified model building,'' \textit{Phys. Rep.}, vol.~79, pp.~1--128, 1981.

\end{thebibliography}

\end{document}
