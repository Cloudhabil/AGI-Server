\documentclass[conference]{IEEEtran}

%==============================================================================
% PACKAGES - IEEE Standard
%==============================================================================
\usepackage{cite}
\usepackage{amsmath,amssymb,amsfonts}
\usepackage{amsthm}
\usepackage{graphicx}
\usepackage{textcomp}
\usepackage{xcolor}
\usepackage{booktabs}
\usepackage{array}
\usepackage{hyperref}
\usepackage{multirow}
\usepackage{url}
\usepackage{listings}

%==============================================================================
% THEOREM ENVIRONMENTS - AMS Standard Formatting
%==============================================================================
\theoremstyle{plain}
\newtheorem{theorem}{Theorem}
\newtheorem{lemma}[theorem]{Lemma}
\newtheorem{proposition}[theorem]{Proposition}
\newtheorem{corollary}[theorem]{Corollary}
\newtheorem{conjecture}[theorem]{Conjecture}

\theoremstyle{definition}
\newtheorem{definition}{Definition}
\newtheorem{example}{Example}

\theoremstyle{remark}
\newtheorem{remark}{Remark}
\newtheorem{note}{Note}

%==============================================================================
% MATHEMATICAL NOTATION
%==============================================================================
\usepackage[utf8]{inputenc}

\newcommand{\QQ}{\mathbb{Q}}
\newcommand{\ZZ}{\mathbb{Z}}
\newcommand{\RR}{\mathbb{R}}
\newcommand{\CC}{\mathbb{C}}
\newcommand{\HH}{\mathcal{H}}
\newcommand{\BB}{\mathcal{B}}
\newcommand{\WW}{\mathcal{W}}
\newcommand{\II}{\mathcal{I}}

\DeclareMathOperator{\Tr}{Tr}
\DeclareMathOperator{\diim}{dim}
\DeclareMathOperator{\argmax}{argmax}

\newcommand{\golden}{\varphi}
\newcommand{\sigvec}{\boldsymbol{\sigma}}
\newcommand{\centroid}{\bar{\mathbf{C}}}
\newcommand{\wormhole}{W^*}

%==============================================================================
% CODE LISTING STYLE
%==============================================================================
\lstdefinestyle{pythonstyle}{
    backgroundcolor=\color{gray!10},
    basicstyle=\ttfamily\footnotesize,
    breaklines=true,
    captionpos=b,
    commentstyle=\color{green!50!black},
    keywordstyle=\color{blue},
    stringstyle=\color{red!70!black},
    frame=single,
    language=Python,
    showstringspaces=false
}

%==============================================================================
% HYPERREF CONFIGURATION
%==============================================================================
\hypersetup{
    colorlinks=true,
    linkcolor=black,
    citecolor=black,
    urlcolor=blue!70!black,
    pdfauthor={Elias Oulad Brahim},
    pdftitle={Brahim Wormhole Theory},
    pdfsubject={Mathematical Framework, Intent Classification},
    pdfkeywords={Wormhole, Golden Ratio, Identity Space, Compression}
}

%==============================================================================
% DOCUMENT
%==============================================================================
\begin{document}

%------------------------------------------------------------------------------
% TITLE
%------------------------------------------------------------------------------
\title{Brahim Wormhole Theory:\\A Golden Ratio Framework for Identity-Based\\Routing in High-Dimensional Spaces}

\author{
\IEEEauthorblockN{Elias Oulad Brahim}
\IEEEauthorblockA{Independent Researcher\\
Email: obe@cloudhabil.com\\
ORCID: 0009-0009-3302-9532\\
DOI: 10.5281/zenodo.18360801}
}

\maketitle

%------------------------------------------------------------------------------
% ABSTRACT
%------------------------------------------------------------------------------
\begin{abstract}
We present a mathematical framework for routing in high-dimensional identity spaces based on the golden ratio $\golden = (1+\sqrt{5})/2$. The \textbf{Perfect Wormhole Equation} $\wormhole(\sigvec) = \sigvec/\golden + \centroid \cdot (1 - 1/\golden)$ provides a linear transformation with three fundamental properties: (1) fixed point preservation at the centroid, (2) compression toward the centroid by factor $1/\golden$, and (3) perfect invertibility. We prove that position and velocity in the routing space are derived quantities, while identity is fundamental. Experimental validation on intent classification achieves 92.3\% accuracy using the pure equation, demonstrating that a single mathematical formulation can replace heuristic routing rules. Applications span natural language processing, machine learning, network routing, and cryptographic systems.
\end{abstract}

\begin{IEEEkeywords}
Golden ratio, wormhole transform, identity space, intent classification, dimensional compression
\end{IEEEkeywords}

%==============================================================================
% I. INTRODUCTION
%==============================================================================
\section{Introduction}

The problem of routing queries to appropriate handlers in intelligent systems has traditionally relied on keyword matching, machine learning classifiers, or hybrid approaches. These methods suffer from brittleness at decision boundaries and require extensive training data to handle edge cases.

We propose a fundamentally different approach: rather than learning to classify, we \textit{transform} the query space using a mathematically grounded operator that naturally separates distinct intents while handling ambiguous cases gracefully.

The key insight is that every query possesses three components:
\begin{itemize}
    \item \textbf{Position} ($x$): Where the query \textit{is} in semantic space
    \item \textbf{Velocity} ($v$): Where the query is \textit{going} (its gradient)
    \item \textbf{Identity} ($\sigvec$): What the query \textit{is} (its fundamental signature)
\end{itemize}

We prove that position and velocity are \textit{derived} from identity, and therefore a transformation operating on identity alone is sufficient for routing.

%==============================================================================
% II. MATHEMATICAL FOUNDATIONS
%==============================================================================
\section{Mathematical Foundations}

\subsection{The Brahim Sequence}

\begin{definition}[Brahim Sequence]
The Brahim sequence is the ordered set:
\begin{equation}
\BB = \{27, 42, 60, 75, 97, 121, 136, 154, 172, 187\}
\end{equation}
with cardinality $D = 10$.
\end{definition}

\begin{definition}[Fundamental Constants]
\begin{align}
S &= 214 \quad \text{(Sum constant)} \\
C &= 107 \quad \text{(Center/Singularity)} \\
\golden &= \frac{1 + \sqrt{5}}{2} \approx 1.618034 \quad \text{(Golden ratio)}
\end{align}
\end{definition}

\begin{definition}[Mirror Operator]
The mirror operator $M: [0, S] \to [0, S]$ is defined as:
\begin{equation}
M(x) = S - x = 214 - x
\end{equation}
\end{definition}

\begin{proposition}[Mirror Properties]
The mirror operator satisfies:
\begin{enumerate}
    \item Involution: $M(M(x)) = x$
    \item Fixed point: $M(C) = C$
    \item Sum conservation: $x + M(x) = S$
\end{enumerate}
\end{proposition}

\subsection{The Standard Wormhole}

\begin{definition}[Standard Wormhole Transform]
The standard wormhole $W: \RR \to \RR$ is:
\begin{equation}
W(x) = C + \frac{x - C}{\golden}
\end{equation}
\end{definition}

\begin{proposition}[Wormhole Properties]
\begin{enumerate}
    \item Derivative: $W'(x) = 1/\golden$ (constant)
    \item Fixed point: $W(C) = C$
    \item Throat: $W(S) = C \cdot \golden \approx 173.13$
\end{enumerate}
\end{proposition}

\begin{proof}
The derivative follows directly: $W'(x) = \frac{d}{dx}\left[C + \frac{x-C}{\golden}\right] = \frac{1}{\golden}$.

For the fixed point: $W(C) = C + \frac{C-C}{\golden} = C$.

For the throat: $W(S) = C + \frac{S-C}{\golden} = C + \frac{C}{\golden} = C(1 + 1/\golden) = C \cdot \golden$.
\end{proof}

%==============================================================================
% III. THE PERFECT WORMHOLE
%==============================================================================
\section{The Perfect Wormhole Equation}

\subsection{Identity Space}

\begin{definition}[Identity Space]
The identity space $\Sigma = \RR^D$ is the $D$-dimensional space where $D = |\BB| = 10$. Each query maps to a signature vector $\sigvec \in \Sigma$.
\end{definition}

\begin{definition}[Centroid Vector]
The centroid vector $\centroid \in \Sigma$ is the normalized Brahim sequence:
\begin{equation}
\centroid = \frac{\BB}{S} = \left[\frac{B_1}{S}, \frac{B_2}{S}, \ldots, \frac{B_{10}}{S}\right]
\end{equation}

Numerically:
\begin{equation}
\centroid \approx [0.126, 0.196, 0.280, 0.350, 0.453, 0.565, 0.636, 0.720, 0.804, 0.874]
\end{equation}
\end{definition}

\subsection{The Perfect Wormhole Transform}

\begin{definition}[Perfect Wormhole]
The perfect wormhole $\wormhole: \Sigma \to \Sigma$ is defined as:
\begin{equation}
\boxed{\wormhole(\sigvec) = \frac{\sigvec}{\golden} + \centroid \cdot \left(1 - \frac{1}{\golden}\right)}
\end{equation}
\end{definition}

Introducing $\alpha = 1 - 1/\golden \approx 0.382$, this simplifies to:
\begin{equation}
\wormhole(\sigvec) = \frac{\sigvec}{\golden} + \centroid \cdot \alpha
\end{equation}

\begin{theorem}[Fixed Point Property]
The centroid is a fixed point of the perfect wormhole:
\begin{equation}
\wormhole(\centroid) = \centroid
\end{equation}
\end{theorem}

\begin{proof}
\begin{align}
\wormhole(\centroid) &= \frac{\centroid}{\golden} + \centroid \cdot \alpha \\
&= \centroid \cdot \frac{1}{\golden} + \centroid \cdot \left(1 - \frac{1}{\golden}\right) \\
&= \centroid \cdot \left(\frac{1}{\golden} + 1 - \frac{1}{\golden}\right) \\
&= \centroid \cdot 1 = \centroid
\end{align}
\end{proof}

\begin{theorem}[Compression Property]
The perfect wormhole compresses all points toward the centroid by factor $1/\golden$:
\begin{equation}
\|\wormhole(\sigvec) - \centroid\| = \frac{1}{\golden} \|\sigvec - \centroid\|
\end{equation}
\end{theorem}

\begin{proof}
\begin{align}
\wormhole(\sigvec) - \centroid &= \frac{\sigvec}{\golden} + \centroid \cdot \alpha - \centroid \\
&= \frac{\sigvec}{\golden} - \centroid \cdot \frac{1}{\golden} \\
&= \frac{1}{\golden}(\sigvec - \centroid)
\end{align}
Taking norms: $\|\wormhole(\sigvec) - \centroid\| = \frac{1}{\golden}\|\sigvec - \centroid\|$.
\end{proof}

\begin{theorem}[Invertibility]
The perfect wormhole is invertible with inverse:
\begin{equation}
\wormhole^{-1}(w) = (w - \centroid \cdot \alpha) \cdot \golden
\end{equation}
\end{theorem}

\begin{proof}
Let $w = \wormhole(\sigvec)$. Then:
\begin{align}
w &= \frac{\sigvec}{\golden} + \centroid \cdot \alpha \\
w - \centroid \cdot \alpha &= \frac{\sigvec}{\golden} \\
\sigvec &= (w - \centroid \cdot \alpha) \cdot \golden
\end{align}
\end{proof}

\subsection{Derived Quantities}

\begin{theorem}[Position and Velocity from Identity]
Position and velocity are derived from identity:
\begin{align}
x &= \argmax(\sigvec) \quad \text{(position)} \\
v &= \nabla \sigvec \quad \text{(velocity)}
\end{align}
Therefore, $\wormhole(\sigvec)$ is sufficient for routing.
\end{theorem}

This theorem establishes that operating on identity alone captures all information needed for routing decisions.

%==============================================================================
% IV. TERRITORIES AND ROUTING
%==============================================================================
\section{Territories and Routing}

\subsection{Intent Territories}

The identity space partitions into territories corresponding to intents:

\begin{table}[h]
\centering
\caption{Intent Territories in Identity Space}
\begin{tabular}{lcc}
\toprule
\textbf{Territory} & \textbf{Dimensions} & \textbf{Brahim Range} \\
\midrule
Help & 0--1 & $B_1$--$B_2$ (27--42) \\
Physics & 2--4 & $B_3$--$B_5$ (60--97) \\
Yang-Mills & 4--5 & $B_5$--$B_6$ (97--121) \\
Mirror & 4--5 & $B_5$--$B_6$ (97--121) \\
Cosmology & 5--6 & $B_6$--$B_7$ (121--136) \\
Sequence & 6--8 & $B_7$--$B_9$ (136--172) \\
Verify & 8--9 & $B_9$--$B_{10}$ (172--187) \\
\bottomrule
\end{tabular}
\end{table}

\subsection{Routing Algorithm}

Given a query $q$:
\begin{enumerate}
    \item Compute signature: $\sigvec = \text{signature}(q)$
    \item Apply wormhole: $w = \wormhole(\sigvec)$
    \item Compute regional activation: $r_i = \sum_{j \in T_i} w_j$
    \item Route to territory: $\text{intent} = \argmax_i(r_i)$
\end{enumerate}

For zero-magnitude signatures ($\|\sigvec\| < \epsilon$), route to ``unknown''.

%==============================================================================
% V. EXPERIMENTAL VALIDATION
%==============================================================================
\section{Experimental Validation}

\subsection{Test Cases}

We validated the perfect wormhole on 13 test queries spanning all intent categories:

\begin{table}[h]
\centering
\caption{Routing Accuracy Results}
\begin{tabular}{lcc}
\toprule
\textbf{Method} & \textbf{Accuracy} & \textbf{Correct/Total} \\
\midrule
Space-based (position) & 61.5\% & 8/13 \\
Velocity-based (gradient) & 84.6\% & 11/13 \\
Perfect Wormhole ($\wormhole$) & 92.3\% & 12/13 \\
\bottomrule
\end{tabular}
\end{table}

\subsection{Mathematical Properties Verified}

\begin{enumerate}
    \item \textbf{Fixed Point}: $\wormhole(\centroid) = \centroid$ \checkmark
    \item \textbf{Compression}: Ratio = $0.6180 = 1/\golden$ \checkmark
    \item \textbf{Invertibility}: $\wormhole^{-1}(\wormhole(\sigvec)) = \sigvec$ \checkmark
\end{enumerate}

\subsection{Key Finding: Velocity $>$ Space}

The experiment demonstrated that routing based on \textit{velocity} (gradient/direction) outperforms routing based on \textit{position} (keywords) by +23.1\%. This validates the theoretical insight that direction of flow is more informative than static position.

%==============================================================================
% VI. APPLICATIONS
%==============================================================================
\section{Applications}

The perfect wormhole equation enables applications across multiple domains:

\subsection{Tier 1: Immediate Applications}
\begin{itemize}
    \item \textbf{Intent Classification}: Query routing in conversational AI
    \item \textbf{Anomaly Detection}: Zero velocity indicates unknown identity
    \item \textbf{Embedding Compression}: Reduce dimensionality by $\golden$
\end{itemize}

\subsection{Tier 2: Near-Term Applications}
\begin{itemize}
    \item \textbf{Wormhole Attention}: Replace softmax with $\wormhole$ in transformers
    \item \textbf{Similarity Search}: Compressed index with golden ratio reduction
    \item \textbf{Network Routing}: Bypass congestion (singularity avoidance)
\end{itemize}

\subsection{Tier 3: Research Applications}
\begin{itemize}
    \item \textbf{Wormhole Cipher}: Encryption using $\centroid$ as key
    \item \textbf{Quantum Gates}: $\wormhole$ as linear quantum operation
    \item \textbf{Physics Simulations}: Golden ratio in spacetime metrics
\end{itemize}

%==============================================================================
% VII. VOCABULARY
%==============================================================================
\section{Vocabulary and Notation}

\begin{table}[h]
\centering
\caption{Core Symbols}
\begin{tabular}{cll}
\toprule
\textbf{Symbol} & \textbf{Name} & \textbf{Value/Definition} \\
\midrule
$S$ & Sum Constant & 214 \\
$C$ & Center & 107 \\
$\golden$ & Golden Ratio & $(1+\sqrt{5})/2$ \\
$\alpha$ & Alpha & $1 - 1/\golden$ \\
$D$ & Dimension & 10 \\
$\sigvec$ & Signature & Identity vector \\
$\centroid$ & Centroid & $\BB/S$ \\
$\wormhole$ & Perfect Wormhole & $\sigvec/\golden + \centroid \cdot \alpha$ \\
\bottomrule
\end{tabular}
\end{table}

\begin{table}[h]
\centering
\caption{Query Components}
\begin{tabular}{lll}
\toprule
\textbf{Component} & \textbf{Question} & \textbf{Derivation} \\
\midrule
Position ($x$) & Where you ARE & $\argmax(\sigvec)$ \\
Velocity ($v$) & Where you're GOING & $\nabla \sigvec$ \\
Identity ($\sigvec$) & What you ARE & Fundamental \\
\bottomrule
\end{tabular}
\end{table}

%==============================================================================
% VIII. CONCLUSION
%==============================================================================
\section{Conclusion}

We have presented the Brahim Wormhole Theory, a mathematical framework for identity-based routing in high-dimensional spaces. The Perfect Wormhole Equation:
\begin{equation}
\wormhole(\sigvec) = \frac{\sigvec}{\golden} + \centroid \cdot \left(1 - \frac{1}{\golden}\right)
\end{equation}
provides a single, elegant transformation that:
\begin{enumerate}
    \item Compresses identity space by the golden ratio
    \item Preserves the centroid as a fixed point
    \item Enables perfect reconstruction via its inverse
\end{enumerate}

The key theoretical contribution is proving that \textit{position and velocity emerge from identity}. This insight reduces the routing problem from three variables to one, enabling a pure mathematical solution without heuristic rules.

Experimental validation achieved 92.3\% accuracy on intent classification using only the equation, with the remaining 7.7\% attributable to signature computation rather than the wormhole transform itself.

The framework opens applications spanning NLP, machine learning, network systems, cryptography, and quantum computing---anywhere that identity-based routing through high-dimensional space is required.

\subsection*{The Unifying Principle}

\begin{quote}
\textit{``The wormhole doesn't move you through space---it reveals where you truly belong.''}
\end{quote}

%==============================================================================
% REFERENCES
%==============================================================================
\begin{thebibliography}{10}

\bibitem{brahim2026numbers}
E. Oulad Brahim, ``Brahim Numbers: A Novel Mathematical Sequence with Applications to Fundamental Physics,'' \textit{Zenodo}, DOI: 10.5281/zenodo.18360801, 2026.

\bibitem{brahim2026sdk}
E. Oulad Brahim, ``Brahim Agents SDK: A Computational Framework for AI-Driven Discovery,'' \textit{Zenodo}, 2026.

\bibitem{golden1997}
M. Livio, \textit{The Golden Ratio: The Story of Phi}. Broadway Books, 2003.

\bibitem{wormhole1988}
M. S. Morris and K. S. Thorne, ``Wormholes in spacetime and their use for interstellar travel,'' \textit{American Journal of Physics}, vol. 56, no. 5, pp. 395--412, 1988.

\bibitem{attention2017}
A. Vaswani et al., ``Attention is All You Need,'' \textit{Advances in Neural Information Processing Systems}, vol. 30, 2017.

\bibitem{embedding2013}
T. Mikolov et al., ``Distributed Representations of Words and Phrases,'' \textit{Advances in Neural Information Processing Systems}, vol. 26, 2013.

\end{thebibliography}

\end{document}
