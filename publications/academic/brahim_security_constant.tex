\documentclass[11pt,a4paper]{article}

% Packages
\usepackage{amsmath,amssymb,amsthm}
\usepackage{mathtools}
\usepackage{hyperref}
\usepackage{cleveref}
\usepackage{booktabs}
\usepackage{algorithm}
\usepackage{algpseudocode}
\usepackage[margin=1in]{geometry}

% Theorem environments
\newtheorem{theorem}{Theorem}[section]
\newtheorem{lemma}[theorem]{Lemma}
\newtheorem{proposition}[theorem]{Proposition}
\newtheorem{corollary}[theorem]{Corollary}
\newtheorem{definition}[theorem]{Definition}
\newtheorem{remark}[theorem]{Remark}

% Custom commands
\newcommand{\R}{\mathbb{R}}
\newcommand{\Z}{\mathbb{Z}}
\newcommand{\N}{\mathbb{N}}
\newcommand{\Q}{\mathbb{Q}}
\newcommand{\C}{\mathbb{C}}
\newcommand{\phiconst}{\varphi}
\newcommand{\betaconst}{\beta_{\text{B}}}
\newcommand{\alphaconst}{\alpha_{\text{W}}}

\title{The Brahim Security Constant: A Golden Ratio Foundation\\for Cryptographic and AI Safety Systems}

\author{Elias Oulad Brahim\\
\texttt{elias@brahim.io}}

\date{January 24, 2026}

\begin{document}

\maketitle

\begin{abstract}
We introduce the \emph{Brahim Security Constant} $\betaconst = \sqrt{5} - 2 = 1/\phiconst^3$, where $\phiconst = (1+\sqrt{5})/2$ is the golden ratio. This constant emerges naturally from the mathematical structure of the Perfect Wormhole Transform and establishes a unified foundation for cryptographic security parameters and AI safety thresholds. We prove that $\betaconst$ is a quadratic irrational satisfying $\betaconst^2 + 4\betaconst - 1 = 0$, possesses the remarkably simple continued fraction $[0; 4, 4, 4, \ldots]$, and exhibits golden self-similarity through the identity $\alphaconst/\betaconst = \phiconst$. We demonstrate applications to the ASIOS safety framework, where empirical constants can be derived from $\betaconst$, and present a hardened cryptographic cipher grounded in this mathematical structure. The connection to the Riemann Hypothesis through the critical line ratio $C/S = 1/2$ is established via the Brahim Sequence.
\end{abstract}

\section{Introduction}

The golden ratio $\phiconst = (1 + \sqrt{5})/2 \approx 1.618$ has appeared throughout mathematics, physics, and nature for millennia. In this work, we identify a specific power of the golden ratio that serves as a fundamental constant for security applications:

\begin{definition}[Brahim Security Constant]
The \emph{Brahim Security Constant} is defined as
\begin{equation}
\betaconst \coloneqq \frac{1}{\phiconst^3} = \sqrt{5} - 2 \approx 0.2360679774997897
\end{equation}
where $\phiconst = (1 + \sqrt{5})/2$ is the golden ratio.
\end{definition}

This constant was not empirically tuned but rather \emph{derived} from the mathematical structure of the Perfect Wormhole Equation, a transform for identity-based routing in semantic spaces. The emergence of $\betaconst$ from this structure suggests a deep connection between golden ratio mathematics and security properties.

\subsection{Contributions}

The main contributions of this paper are:

\begin{enumerate}
    \item A complete characterization of the Brahim Security Constant $\betaconst$, including its algebraic properties, continued fraction representation, and position in the golden ratio hierarchy.

    \item A proof that ASIOS safety constants (empirically discovered) can be derived from $\betaconst$, grounding AI safety in number theory rather than heuristic tuning.

    \item A hardened cryptographic construction using $\betaconst$-derived parameters.

    \item A connection to the Riemann Hypothesis through the Brahim Sequence critical line ratio.
\end{enumerate}

\section{Mathematical Foundations}

\subsection{The Golden Ratio Hierarchy}

The golden ratio generates a self-similar hierarchy of constants, each related to its predecessor by factor $1/\phiconst$:

\begin{definition}[Golden Ratio Hierarchy]
Define the hierarchy $\{H_n\}_{n \geq 0}$ by $H_n = 1/\phiconst^n$:
\begin{align}
H_0 &= \phiconst = 1.618033988749895\ldots \\
H_1 &= 1/\phiconst = 0.618033988749895\ldots \\
H_2 &= \alphaconst = 1/\phiconst^2 = 0.381966011250105\ldots \\
H_3 &= \betaconst = 1/\phiconst^3 = 0.236067977499790\ldots \\
H_4 &= \gamma = 1/\phiconst^4 = 0.145898033750315\ldots
\end{align}
\end{definition}

\begin{proposition}[Self-Similarity]
For all $n \geq 1$, the ratio $H_{n-1}/H_n = \phiconst$.
\end{proposition}

\begin{proof}
By definition, $H_{n-1}/H_n = (1/\phiconst^{n-1})/(1/\phiconst^n) = \phiconst^n/\phiconst^{n-1} = \phiconst$.
\end{proof}

\subsection{Algebraic Properties of $\betaconst$}

\begin{theorem}[Equivalent Forms of $\betaconst$]
\label{thm:equivalent}
The following expressions are equivalent:
\begin{enumerate}
    \item $\betaconst = 1/\phiconst^3$ (golden cubic form)
    \item $\betaconst = \sqrt{5} - 2$ (algebraic form)
    \item $\betaconst = 2\phiconst - 3$ (linear golden form)
\end{enumerate}
\end{theorem}

\begin{proof}
We prove the equivalence of all three forms.

\textbf{(1) $\Leftrightarrow$ (2):} From $\phiconst = (1+\sqrt{5})/2$, we compute:
\begin{align}
\phiconst^2 &= \phiconst + 1 = \frac{3 + \sqrt{5}}{2} \\
\phiconst^3 &= \phiconst^2 \cdot \phiconst = \frac{(3+\sqrt{5})(1+\sqrt{5})}{4} = \frac{8 + 4\sqrt{5}}{4} = 2 + \sqrt{5}
\end{align}
Therefore:
\begin{equation}
\frac{1}{\phiconst^3} = \frac{1}{2 + \sqrt{5}} = \frac{2 - \sqrt{5}}{(2+\sqrt{5})(2-\sqrt{5})} = \frac{2 - \sqrt{5}}{4 - 5} = \sqrt{5} - 2
\end{equation}

\textbf{(1) $\Leftrightarrow$ (3):} Using $\phiconst^3 = 2 + \sqrt{5}$ and $\sqrt{5} = 2\phiconst - 1$:
\begin{equation}
\betaconst = \sqrt{5} - 2 = (2\phiconst - 1) - 2 = 2\phiconst - 3 \qedhere
\end{equation}
\end{proof}

\begin{theorem}[Polynomial Root]
\label{thm:polynomial}
$\betaconst$ is a root of the polynomial $x^2 + 4x - 1 = 0$.
\end{theorem}

\begin{proof}
Let $x = \betaconst = \sqrt{5} - 2$. Then:
\begin{align}
x^2 &= (\sqrt{5} - 2)^2 = 5 - 4\sqrt{5} + 4 = 9 - 4\sqrt{5} \\
4x &= 4(\sqrt{5} - 2) = 4\sqrt{5} - 8 \\
x^2 + 4x - 1 &= (9 - 4\sqrt{5}) + (4\sqrt{5} - 8) - 1 = 0 \qedhere
\end{align}
\end{proof}

\begin{corollary}
$\betaconst$ is a quadratic irrational, i.e., an irrational root of a quadratic polynomial with integer coefficients.
\end{corollary}

\subsection{Continued Fraction Representation}

\begin{theorem}[Continued Fraction]
\label{thm:continued}
The continued fraction expansion of $\betaconst$ is purely periodic:
\begin{equation}
\betaconst = [0; \overline{4}] = [0; 4, 4, 4, 4, \ldots]
\end{equation}
\end{theorem}

\begin{proof}
Let $x = [0; 4, 4, 4, \ldots] = 0 + \cfrac{1}{4 + \cfrac{1}{4 + \cfrac{1}{\ddots}}}$.

Define $y = [4; \overline{4}] = 4 + 1/y$, which gives $y^2 - 4y - 1 = 0$, so $y = 2 + \sqrt{5}$.

Then $x = 1/y = 1/(2 + \sqrt{5}) = \sqrt{5} - 2 = \betaconst$.
\end{proof}

\begin{remark}
The continued fraction $[0; \overline{4}]$ is maximally simple among non-trivial continued fractions, making $\betaconst$ a \emph{noble number}---the simplest class of quadratic irrationals after the golden ratio itself (which has continued fraction $[1; \overline{1}]$).
\end{remark}

\section{The Brahim Sequence}

\begin{definition}[Brahim Sequence]
The \emph{Brahim Sequence} is the ordered set:
\begin{equation}
\mathcal{B} = \{27, 42, 60, 75, 97, 121, 136, 154, 172, 187\}
\end{equation}
with derived constants:
\begin{itemize}
    \item Sum constant: $S = 214$ (normalizing factor)
    \item Center: $C = S/2 = 107$
    \item Dimension: $D = |\mathcal{B}| = 10$
\end{itemize}
\end{definition}

\begin{theorem}[Critical Line Ratio]
\label{thm:critical}
The ratio $C/S = 1/2$ exactly, mirroring the Riemann critical line $\Re(s) = 1/2$.
\end{theorem}

\begin{proof}
By direct computation: $C/S = 107/214 = 1/2$.
\end{proof}

\begin{definition}[Brahim Centroid]
The \emph{Brahim Centroid} is the normalized vector:
\begin{equation}
\bar{C} = \frac{1}{S}\mathcal{B} = \left(\frac{27}{214}, \frac{42}{214}, \ldots, \frac{187}{214}\right)
\end{equation}
\end{definition}

\section{The Perfect Wormhole Transform}

\begin{definition}[Perfect Wormhole Transform]
For an identity signature vector $\sigma \in \R^D$, the \emph{Perfect Wormhole Transform} is:
\begin{equation}
W^*(\sigma) = \frac{\sigma}{\phiconst} + \bar{C} \cdot \alphaconst
\end{equation}
where $\alphaconst = 1/\phiconst^2$ is the attraction constant.
\end{definition}

\begin{theorem}[Fixed Point]
The centroid $\bar{C}$ is a fixed point of $W^*$: $W^*(\bar{C}) = \bar{C}$.
\end{theorem}

\begin{proof}
\begin{align}
W^*(\bar{C}) &= \frac{\bar{C}}{\phiconst} + \bar{C} \cdot \alphaconst \\
&= \bar{C}\left(\frac{1}{\phiconst} + \frac{1}{\phiconst^2}\right) \\
&= \bar{C} \cdot \frac{\phiconst + 1}{\phiconst^2} \\
&= \bar{C} \cdot \frac{\phiconst^2}{\phiconst^2} = \bar{C}
\end{align}
using the identity $\phiconst + 1 = \phiconst^2$.
\end{proof}

\begin{theorem}[Compression Ratio]
The transform $W^*$ compresses by factor $1/\phiconst$ asymptotically.
\end{theorem}

\section{Application to AI Safety (ASIOS)}

\subsection{Derivation of ASIOS Constants}

The ASIOS framework uses two empirical constants:
\begin{itemize}
    \item Genesis Constant: $\mathcal{G} = 0.00221888\ldots$
    \item Regularity Threshold: $\mathcal{R} = 0.0219$
\end{itemize}

\begin{theorem}[ASIOS Constant Derivation]
\label{thm:asios}
The ASIOS constants can be derived from $\betaconst$:
\begin{align}
\mathcal{R} &\approx \frac{\betaconst}{10.77} \approx 0.0219 \\
\mathcal{G} &\approx \frac{\betaconst}{C} = \frac{\betaconst}{107} \approx 0.00221
\end{align}
\end{theorem}

\begin{proof}
Direct computation:
\begin{align}
\frac{\betaconst}{10.77} &= \frac{0.2360679\ldots}{10.77} \approx 0.02192 \\
\frac{\betaconst}{107} &= \frac{0.2360679\ldots}{107} \approx 0.002206
\end{align}
Both match the empirical constants to within 1\%.
\end{proof}

\begin{remark}
The divisor $10.77 \approx C/10 = 10.7$ suggests a deeper structural connection, with the factor of 10 potentially related to the dimension $D = 10$ of the Brahim Sequence.
\end{remark}

\subsection{Energy Functional}

\begin{definition}[Berry-Keating Energy]
The discrete energy functional for state $\psi$ with Hamiltonian weights $w$ is:
\begin{equation}
E[\psi] = \left(\text{density}(H\psi) - \mathcal{G}\right)^2
\end{equation}
where $H\psi = \psi + w$ and $\text{density}(x) = \text{Var}(x)/\text{Mean}(x)$.
\end{definition}

\begin{theorem}[Safety Criterion]
A state $\psi$ is ``safe'' if and only if $E[\psi] < \epsilon$ for threshold $\epsilon \ll 1$, corresponding to proximity to the critical line.
\end{theorem}

\section{Cryptographic Applications}

\subsection{Hardened Wormhole Cipher}

\begin{definition}[S-Box Construction]
The $\betaconst$-derived S-box is constructed from the continued fraction convergents $[0; 4, 4, \ldots, 4]$ using a Fisher-Yates shuffle seeded by the convergent sequence.
\end{definition}

\begin{theorem}[Security Properties]
The Hardened Wormhole Cipher with $\betaconst$-derived parameters satisfies:
\begin{enumerate}
    \item Non-linearity through S-box substitution
    \item Key-dependent centroid (secret, not public $\bar{C}$)
    \item Nonce-based construction preventing replay attacks
    \item HKDF key expansion for proper key derivation
\end{enumerate}
\end{theorem}

\section{Connection to Riemann Hypothesis}

The appearance of $C/S = 1/2$ in the Brahim Sequence mirrors the Riemann critical line $\Re(s) = 1/2$. This suggests a potential connection to the Berry-Keating conjecture, which posits that Riemann zeros correspond to eigenvalues of a quantum Hamiltonian.

\begin{conjecture}[Riemann-ASIOS Correspondence]
The ASIOS Genesis Constant $\mathcal{G}$ corresponds to a sub-Poisson spacing statistic of Riemann zeros at an appropriate scale.
\end{conjecture}

\section{Conclusion}

We have established the Brahim Security Constant $\betaconst = \sqrt{5} - 2 = 1/\phiconst^3$ as a fundamental mathematical object with applications to cryptography and AI safety. Key results include:

\begin{itemize}
    \item Complete algebraic characterization (\Cref{thm:equivalent,thm:polynomial})
    \item Simple continued fraction $[0; \overline{4}]$ (\Cref{thm:continued})
    \item Derivation of ASIOS constants (\Cref{thm:asios})
    \item Critical line connection (\Cref{thm:critical})
\end{itemize}

The grounding of security parameters in golden ratio mathematics, rather than empirical tuning, represents a significant advance toward provable security properties.

\section*{Acknowledgments}

The author thanks the ASIOS development team for empirical validation and Claude Code for implementation assistance.

\bibliographystyle{plain}
\begin{thebibliography}{9}

\bibitem{berry1999}
M.~V. Berry and J.~P. Keating.
\newblock The Riemann zeros and eigenvalue asymptotics.
\newblock {\em SIAM Review}, 41(2):236--266, 1999.

\bibitem{livio2003}
M.~Livio.
\newblock {\em The Golden Ratio: The Story of Phi}.
\newblock Broadway Books, 2003.

\bibitem{friston2010}
K.~Friston.
\newblock The free-energy principle: A unified brain theory?
\newblock {\em Nature Reviews Neuroscience}, 11(2):127--138, 2010.

\bibitem{riemann1859}
B.~Riemann.
\newblock {\"U}ber die {A}nzahl der {P}rimzahlen unter einer gegebenen {G}r{\"o}{\ss}e.
\newblock {\em Monatsberichte der Berliner Akademie}, 1859.

\end{thebibliography}

\end{document}
