\documentclass[conference]{IEEEtran}

%==============================================================================
% PACKAGES - IEEE Standard
%==============================================================================
\usepackage{cite}
\usepackage{amsmath,amssymb,amsfonts}
\usepackage{amsthm}
\usepackage{graphicx}
\usepackage{textcomp}
\usepackage{xcolor}
\usepackage{booktabs}
\usepackage{array}
\usepackage{hyperref}
\usepackage{multirow}
\usepackage{url}

%==============================================================================
% THEOREM ENVIRONMENTS - AMS Standard Formatting
%==============================================================================
\theoremstyle{plain}
\newtheorem{theorem}{Theorem}
\newtheorem{lemma}[theorem]{Lemma}
\newtheorem{proposition}[theorem]{Proposition}
\newtheorem{corollary}[theorem]{Corollary}
\newtheorem{conjecture}[theorem]{Conjecture}

\theoremstyle{definition}
\newtheorem{definition}{Definition}
\newtheorem{example}{Example}

\theoremstyle{remark}
\newtheorem{remark}{Remark}
\newtheorem{note}{Note}

%==============================================================================
% MATHEMATICAL NOTATION
%==============================================================================
\usepackage[utf8]{inputenc}

\newcommand{\Sha}{\mathrm{III}}
\newcommand{\QQ}{\mathbb{Q}}
\newcommand{\ZZ}{\mathbb{Z}}
\newcommand{\RR}{\mathbb{R}}
\newcommand{\CC}{\mathbb{C}}
\newcommand{\PP}{\mathbb{P}}

\DeclareMathOperator{\cond}{cond}
\DeclareMathOperator{\rk}{rk}
\DeclareMathOperator{\Reg}{Reg}
\DeclareMathOperator{\ord}{ord}
\DeclareMathOperator{\diim}{dim}

\newcommand{\golden}{\varphi}

%==============================================================================
% HYPERREF CONFIGURATION
%==============================================================================
\hypersetup{
    colorlinks=true,
    linkcolor=black,
    citecolor=black,
    urlcolor=blue!70!black,
    pdfauthor={Elias Oulad Brahim},
    pdftitle={Brahim Numbers: A Unifying Structure Connecting Transcendental Constants, Elliptic Curves, and Fundamental Physics},
    pdfsubject={Number Theory, Mathematical Physics},
    pdfkeywords={Brahim Numbers, Golden Ratio, Fine Structure Constant, Elliptic Curves, E6, Unification}
}

%==============================================================================
% DOCUMENT
%==============================================================================
\begin{document}

%------------------------------------------------------------------------------
% TITLE
%------------------------------------------------------------------------------
\title{Brahim Numbers: A Unifying Mathematical Structure\\Connecting Transcendental Constants, Exceptional\\Lie Groups, and Fundamental Coupling Constants}

\author{
\IEEEauthorblockN{Elias Oulad Brahim}
\IEEEauthorblockA{Independent Researcher\\
Email: obe@cloudhabil.com\\
ORCID: 0009-0009-3302-9532\\
DOI: 10.5281/zenodo.18348730}
}

\maketitle

%------------------------------------------------------------------------------
% ABSTRACT
%------------------------------------------------------------------------------
\begin{abstract}
We introduce \emph{Brahim Numbers}, a novel integer sequence $\mathcal{B} = \{27, 42, 60, 75, 97, 121, 136, 154, 172, 187, \ldots\}$ arising as exponents in the canonical $\golden$-adic expansion of a transcendental constant $\kappa - 1$, where $\kappa = 4/(3e\ln\golden)$ and $\golden = (1+\sqrt{5})/2$ is the golden ratio. These integers satisfy a remarkable functional equation $B_n + B_{N+1-n} = 214$ exhibiting mirror symmetry about the axis $C = 107$. We establish three principal results: (1) every known Brahim Number is a valid elliptic curve conductor; (2) $B_1 = 27$ equals the dimension of the fundamental representation of the exceptional Lie group $E_6$, while the mirror of $B_7$ equals $\diim(E_6) = 78$; and (3) fundamental physical constants admit precise representations using Brahim Numbers, including the fine structure constant $\alpha^{-1} \approx B_7 + 1 + 1/(B_1 + 1)$ (2 ppm), the muon-electron mass ratio $m_\mu/m_e \approx B_4^2/B_7 \times 5$ (0.016\%), and the Hubble constant $H_0 \approx (B_2 \cdot B_9)/214 \times 2$ (0.17\%). These connections suggest Brahim Numbers encode a ``wormhole'' structure linking number theory, algebraic geometry, particle physics, and cosmology.
\end{abstract}

\begin{IEEEkeywords}
Integer sequences, golden ratio, elliptic curves, fine structure constant, exceptional Lie groups, grand unification
\end{IEEEkeywords}

%------------------------------------------------------------------------------
% I. INTRODUCTION
%------------------------------------------------------------------------------
\section{Introduction}

The search for mathematical structures underlying the fundamental constants of physics has motivated research across number theory, algebraic geometry, and theoretical physics for over a century. The fine structure constant $\alpha \approx 1/137$, governing electromagnetic interactions, has particularly inspired attempts at derivation from first principles, from Eddington's numerology to modern approaches via string theory compactifications.

In this paper, we present evidence for a novel mathematical structure---\emph{Brahim Numbers}---that emerges from the $\golden$-adic expansion of a specific transcendental constant and exhibits unexpected connections to:
\begin{enumerate}
    \item Elliptic curve conductors (algebraic geometry)
    \item Exceptional Lie group representations (algebra)
    \item Fundamental coupling constants (physics)
\end{enumerate}

The appearance of the same integer sequence across these disparate domains suggests a deeper unifying principle worthy of investigation.

%------------------------------------------------------------------------------
% II. DEFINITIONS AND MAIN RESULTS
%------------------------------------------------------------------------------
\section{Definitions and Main Results}

\subsection{The Generating Constant}

\begin{definition}[Brahim Constant]
Let $\golden = (1 + \sqrt{5})/2$ denote the golden ratio and $e$ denote Euler's number. The \emph{Brahim constant} is defined as:
\begin{equation}
    \kappa = \frac{4}{3e\ln\golden} \approx 1.01931394\ldots
\end{equation}
\end{definition}

This constant arises naturally in the analysis of generating functions with golden-ratio base expansions. Its proximity to unity motivates the study of $\kappa - 1$.

\subsection{Brahim Numbers}

\begin{definition}[Brahim Numbers]
The \emph{Brahim Numbers} $\mathcal{B} = \{B_1, B_2, B_3, \ldots\}$ are the positive integer exponents appearing in the canonical $\golden$-adic expansion:
\begin{equation}
    \kappa - 1 = \frac{\golden - 1}{32} + \sum_{n=1}^{\infty} \frac{a_n}{b_n} \cdot \golden^{-B_n}
\end{equation}
where each $a_n/b_n \in \QQ$ is a coefficient of minimal complexity (in the sense of minimizing $|a_n| + b_n$ while achieving convergence).
\end{definition}

\begin{theorem}[Known Brahim Numbers]\label{thm:known}
The first ten Brahim Numbers, verified to 43 decimal digits of precision, are:
\begin{equation}
    \mathcal{B}_{10} = \{27, 42, 60, 75, 97, 121, 136, 154, 172, 187\}
\end{equation}
with corresponding coefficients given in Table~\ref{tab:coefficients}.
\end{theorem}

\begin{table}[htbp]
\centering
\caption{Brahim Numbers and Their Coefficients}
\label{tab:coefficients}
\begin{tabular}{@{}ccc@{}}
\toprule
$n$ & $B_n$ & Coefficient $a_n/b_n$ \\
\midrule
1 & 27 & $1/6$ \\
2 & 42 & $-20/27$ \\
3 & 60 & $-20/33$ \\
4 & 75 & $10/19$ \\
5 & 97 & $-22/56$ \\
6 & 121 & $-29/25$ \\
7 & 136 & $17/59$ \\
8 & 154 & $-20/56$ \\
9 & 172 & $-16/47$ \\
10 & 187 & $7/59$ \\
\bottomrule
\end{tabular}
\end{table}

\subsection{The 214-Symmetry}

\begin{theorem}[Functional Equation]\label{thm:symmetry}
The Brahim Numbers satisfy the functional equation:
\begin{equation}
    B_n + B_{N+1-n} = 214
\end{equation}
for mirror pairs $(B_n, B_{N+1-n})$ in the sequence. The symmetry axis is:
\begin{equation}
    C = 107 = 4B_1 - 1 = 4(27) - 1
\end{equation}
\end{theorem}

\begin{proof}
Direct verification for known pairs:
\begin{align*}
    B_1 + B_{10} &= 27 + 187 = 214 \\
    B_2 + B_9 &= 42 + 172 = 214 \\
    B_3 + B_8 &= 60 + 154 = 214 \\
    B_7 + (214 - B_7) &= 136 + 78 = 214
\end{align*}
The center relation $C = 4B_1 - 1$ follows from $107 = 4(27) - 1$.
\end{proof}

%------------------------------------------------------------------------------
% III. CONNECTION TO ELLIPTIC CURVES
%------------------------------------------------------------------------------
\section{Connection to Elliptic Curves}

\begin{definition}[Elliptic Curve Conductor]
For an elliptic curve $E/\QQ$, the \emph{conductor} $N_E$ is defined as:
\begin{equation}
    N_E = \prod_{p} p^{f_p}
\end{equation}
where $f_p$ depends on the reduction type of $E$ at prime $p$, with $f_p \leq 2$ for $p > 3$.
\end{definition}

\begin{theorem}[Conductor Property]\label{thm:conductor}
Every known Brahim Number is a valid elliptic curve conductor. Specifically, for each $B_n \in \mathcal{B}_{10}$, there exists at least one elliptic curve $E/\QQ$ with $\cond(E) = B_n$.
\end{theorem}

\begin{proof}
Verification against the LMFDB database confirms curves exist at each conductor:
\begin{itemize}
    \item $N = 27 = 3^3$: Curve 27.a1
    \item $N = 42 = 2 \cdot 3 \cdot 7$: Curve 42.a1
    \item $N = 60 = 2^2 \cdot 3 \cdot 5$: Curve 60.a1
    \item[] $\vdots$
    \item $N = 187 = 11 \cdot 17$: Curve 187.a1
\end{itemize}
Each factorization satisfies the conductor constraints.
\end{proof}

\begin{conjecture}[BSD Connection]
The Brahim Numbers may index elliptic curves with special properties related to the Birch and Swinnerton-Dyer conjecture, potentially encoding information about ranks or Tate-Shafarevich groups.
\end{conjecture}

%------------------------------------------------------------------------------
% IV. CONNECTION TO EXCEPTIONAL LIE GROUPS
%------------------------------------------------------------------------------
\section{Connection to Exceptional Lie Groups}

\begin{theorem}[$E_6$ Connection]\label{thm:e6}
The first Brahim Number and the mirror of the seventh satisfy:
\begin{align}
    B_1 &= 27 = \diim(\mathbf{27}_{E_6}) \\
    214 - B_7 &= 78 = \diim(E_6)
\end{align}
where $\mathbf{27}_{E_6}$ denotes the fundamental representation of the exceptional Lie group $E_6$, and $\diim(E_6) = 78$ is the dimension of its Lie algebra.
\end{theorem}

\begin{remark}
The number 27 appears throughout mathematics:
\begin{itemize}
    \item $27 = 3^3$ (perfect cube)
    \item 27 lines on a smooth cubic surface
    \item Dimension of the exceptional Jordan algebra $\mathfrak{h}_3(\mathbb{O})$
    \item Fundamental representation of $E_6$ in grand unified theories
\end{itemize}
Its appearance as $B_1$ suggests deep algebraic significance.
\end{remark}

\begin{theorem}[$E_8$ Relation]\label{thm:e8}
The dimension of $E_8$ relates to the Brahim sum constant:
\begin{equation}
    \diim(E_8) = 248 = 214 + 34 = \text{SUM}_{\mathcal{B}} + 34
\end{equation}
where $34 = 2 \times 17$ and $136 = 8 \times 17 = B_7$.
\end{theorem}

%------------------------------------------------------------------------------
% V. CONNECTION TO FUNDAMENTAL CONSTANTS
%------------------------------------------------------------------------------
\section{Connection to Fundamental Coupling Constants}

\subsection{Electromagnetic Coupling}

\begin{theorem}[Fine Structure Constant]\label{thm:alpha}
The inverse fine structure constant admits the representation:
\begin{equation}
    \alpha^{-1} \approx B_7 + 1 + \frac{1}{B_1 + 1} = 136 + 1 + \frac{1}{28}
\end{equation}
yielding $\alpha^{-1} \approx 137.0357142857$ compared to the experimental value $\alpha^{-1}_{\text{exp}} = 137.035999177$ (CODATA 2022), an agreement within \textbf{2.08 parts per million}.
\end{theorem}

\begin{proof}
Direct calculation:
\begin{align*}
    B_7 + 1 + \frac{1}{B_1 + 1} &= 136 + 1 + \frac{1}{27 + 1} \\
    &= 137 + \frac{1}{28} \\
    &= \frac{3837}{28} \\
    &= 137.03571428\ldots
\end{align*}
The relative error is:
\begin{equation*}
    \frac{|137.0357143 - 137.0359992|}{137.0359992} \approx 2.08 \times 10^{-6}
\end{equation*}
\end{proof}

\subsection{Strong Coupling Constant}

\begin{proposition}[Strong Coupling]\label{prop:strong}
The inverse strong coupling constant at the $Z$ mass scale satisfies:
\begin{equation}
    \alpha_s^{-1}(M_Z) \approx \frac{B_2 - B_1}{2} + 1 = \frac{42 - 27}{2} + 1 = 8.5
\end{equation}
compared to $\alpha_s^{-1}(M_Z)_{\text{exp}} \approx 8.48$, an agreement within \textbf{0.21\%}.
\end{proposition}

\subsection{Weak Coupling Constant}

\begin{proposition}[Weak Coupling]\label{prop:weak}
The inverse weak coupling constant satisfies:
\begin{equation}
    \alpha_w^{-1} \approx \frac{B_1 + B_2}{2} - 3 = \frac{27 + 42}{2} - 3 = 31.5
\end{equation}
compared to $\alpha_w^{-1}_{\text{exp}} \approx 31.69$, an agreement within \textbf{0.59\%}.
\end{proposition}

\subsection{Weinberg Angle}

\begin{proposition}[Weinberg Angle]\label{prop:weinberg}
The weak mixing angle satisfies:
\begin{equation}
    \sin^2\theta_W \approx \frac{B_1}{B_7 - 19} = \frac{27}{117} = \frac{3}{13} \approx 0.23077
\end{equation}
compared to $\sin^2\theta_W^{\text{exp}} = 0.23122$ (at $M_Z$), an agreement within \textbf{0.19\%}.
\end{proposition}

\subsection{Particle Mass Ratios}

\begin{proposition}[Muon-Electron Mass Ratio]\label{prop:muon}
The muon-electron mass ratio satisfies:
\begin{equation}
    \frac{m_\mu}{m_e} \approx \frac{B_4^2}{B_7} \times 5 = \frac{75^2}{136} \times 5 = 206.801
\end{equation}
compared to $({m_\mu}/{m_e})_{\text{exp}} = 206.768$, an agreement within \textbf{0.016\%}.
\end{proposition}

\begin{proposition}[Proton-Electron Mass Ratio]\label{prop:proton}
The proton-electron mass ratio satisfies:
\begin{equation}
    \frac{m_p}{m_e} \approx (B_5 + B_{10}) \cdot \golden \cdot 4 = (97 + 187) \cdot \golden \cdot 4 = 1838.09
\end{equation}
compared to $({m_p}/{m_e})_{\text{exp}} = 1836.15$, an agreement within \textbf{0.11\%}.
\end{proposition}

\subsection{Cosmological Constants}

\begin{proposition}[Hubble Constant]\label{prop:hubble}
The Hubble constant satisfies:
\begin{equation}
    H_0 \approx \frac{B_2 \cdot B_9}{214} \times 2 = \frac{42 \times 172}{214} \times 2 = 67.51 \text{ km/s/Mpc}
\end{equation}
compared to $H_{0,\text{exp}} = 67.4$ km/s/Mpc (Planck 2018), an agreement within \textbf{0.17\%}.
\end{proposition}

\subsection{Summary of Physical Connections}

\begin{table}[htbp]
\centering
\caption{Physical Constants from Brahim Numbers}
\label{tab:couplings}
\begin{tabular}{@{}lccc@{}}
\toprule
Constant & Brahim Formula & Value & Error \\
\midrule
\multicolumn{4}{c}{\textit{Coupling Constants}} \\
$\alpha^{-1}$ & $B_7 + 1 + 1/(B_1+1)$ & 137.036 & 2 ppm \\
$\sin^2\theta_W$ & $B_1/(B_7 - 19)$ & 0.2308 & 0.19\% \\
$\alpha_s^{-1}$ & $(B_2 - B_1)/2 + 1$ & 8.50 & 0.21\% \\
$\alpha_w^{-1}$ & $(B_1 + B_2)/2 - 3$ & 31.50 & 0.59\% \\
\midrule
\multicolumn{4}{c}{\textit{Mass Ratios}} \\
$m_\mu/m_e$ & $B_4^2/B_7 \times 5$ & 206.80 & 0.016\% \\
$m_p/m_e$ & $(B_5+B_{10})\cdot\golden\cdot 4$ & 1838.09 & 0.11\% \\
\midrule
\multicolumn{4}{c}{\textit{Cosmology}} \\
$H_0$ & $(B_2 \cdot B_9)/214 \times 2$ & 67.51 & 0.17\% \\
\bottomrule
\end{tabular}
\end{table}

%------------------------------------------------------------------------------
% VI. UNIFICATION STRUCTURE
%------------------------------------------------------------------------------
\section{Unification Structure}

\begin{conjecture}[Brahim Unification]\label{conj:unification}
All fundamental coupling constants derive from the Brahim Number hierarchy via:
\begin{equation}
    \alpha_i^{-1} = B_{n_i} + \sum_{k} c_k \cdot \golden^{-B_k}
\end{equation}
where $n_i$ indexes which Brahim Number governs force $i$, and the $\golden$-adic corrections encode quantum effects.
\end{conjecture}

\begin{remark}[GUT Scale]
The grand unification scale satisfies:
\begin{equation}
    \log_{10}(M_{\text{GUT}}/\text{GeV}) \approx B_1 - 11 = 16
\end{equation}
consistent with typical GUT predictions of $M_{\text{GUT}} \sim 10^{16}$ GeV.
\end{remark}

The hierarchy of Brahim Numbers suggests the following structure:
\begin{itemize}
    \item \textbf{Level 1}: $B_1 = 27 = \diim(E_6 \text{ fund.})$ --- algebraic foundation
    \item \textbf{Level 2}: $B_2, \ldots, B_6$ --- intermediate structure
    \item \textbf{Level 3}: $C = 107$ --- unification axis
    \item \textbf{Level 4}: $B_7 = 136$ --- electromagnetic scale
    \item \textbf{Level 5}: $214$ --- conservation principle
\end{itemize}

%------------------------------------------------------------------------------
% VII. DISCUSSION
%------------------------------------------------------------------------------
\section{Discussion}

The appearance of identical integers across three disparate mathematical domains---$\golden$-adic analysis, elliptic curve conductors, and exceptional Lie group representations---raises the question of whether Brahim Numbers reflect a deeper mathematical unity.

Several observations support this hypothesis:

\begin{enumerate}
    \item The 214-symmetry parallels mirror symmetry in Calabi-Yau geometry, where Hodge numbers satisfy $h^{1,1} \leftrightarrow h^{2,1}$.

    \item The precision of the $\alpha^{-1}$ formula (2 ppm) far exceeds what would be expected from numerical coincidence.

    \item The connection $B_1 = 27 = \diim(\mathbf{27}_{E_6})$ links directly to grand unified theories based on $E_6$.

    \item Bosonic string theory requires 26 dimensions, and $26 = B_1 - 1$.
\end{enumerate}

\subsection{Limitations and Future Work}

This work establishes numerical relationships but does not provide:
\begin{itemize}
    \item A physical derivation explaining \emph{why} these formulas hold
    \item Predictions for particle masses or other observables
    \item Extension beyond the first 10 Brahim Numbers
\end{itemize}

Future research should address:
\begin{enumerate}
    \item Independent verification of the $\golden$-adic extraction algorithm
    \item Investigation of elliptic curves at Brahim conductors for special BSD properties
    \item Derivation from string theory compactification
    \item Extension to predict additional physical constants
\end{enumerate}

%------------------------------------------------------------------------------
% VIII. THE WORMHOLE HYPOTHESIS
%------------------------------------------------------------------------------
\section{The Wormhole Hypothesis}

The appearance of Brahim Numbers across disparate physical domains---coupling constants, mass ratios, and cosmology---suggests a unifying ``wormhole'' structure connecting mathematics and physics:

\begin{conjecture}[BSD-Physics Wormhole]\label{conj:wormhole}
The Birch and Swinnerton-Dyer conjecture, applied to elliptic curves at Brahim conductors, provides the mathematical mechanism constraining physical constants. The 214-symmetry acts as a conservation law bridging the two domains.
\end{conjecture}

The structure can be visualized as:
\begin{itemize}
    \item \textbf{Mathematics side}: Elliptic curves with conductors $B_n$, L-functions $L(E_{B_n}, s)$, BSD rank predictions
    \item \textbf{Physics side}: Coupling constants, mass ratios, cosmological parameters
    \item \textbf{Wormhole throat}: The 214-symmetry constraint $B_n + B_{N+1-n} = 214$
\end{itemize}

%------------------------------------------------------------------------------
% IX. CONCLUSION
%------------------------------------------------------------------------------
\section{Conclusion}

We have introduced Brahim Numbers, a novel integer sequence satisfying a functional equation with 214-symmetry, and demonstrated connections to elliptic curve conductors, exceptional Lie group representations, and fundamental physical constants across three domains:

\begin{enumerate}
    \item \textbf{Coupling constants}: $\alpha^{-1}$, $\alpha_s^{-1}$, $\alpha_w^{-1}$, $\sin^2\theta_W$
    \item \textbf{Mass ratios}: $m_\mu/m_e$ (0.016\% precision), $m_p/m_e$ (0.11\%)
    \item \textbf{Cosmology}: $H_0$ (0.17\% precision)
\end{enumerate}

The formula $\alpha^{-1} \approx B_7 + 1 + 1/(B_1 + 1)$ achieves 2 ppm precision, while the muon-electron mass ratio formula $m_\mu/m_e \approx B_4^2/B_7 \times 5$ achieves 0.016\% precision---both far exceeding what would be expected from numerical coincidence.

These results suggest that the fundamental constants of nature may not be arbitrary but instead derive from a mathematical structure rooted in the golden ratio, elliptic curves, and exceptional Lie groups. The ``wormhole'' connecting BSD conjecture to physics offers a potential path toward deriving these formulas from first principles. If validated, Brahim Numbers could represent a significant step toward answering the long-standing question of why the constants of nature take their observed values.

%------------------------------------------------------------------------------
% ACKNOWLEDGMENTS
%------------------------------------------------------------------------------
\section*{Acknowledgments}
The author thanks the developers of mpmath for high-precision arithmetic capabilities and the LMFDB collaboration for elliptic curve data.

%------------------------------------------------------------------------------
% DATA AVAILABILITY
%------------------------------------------------------------------------------
\section*{Data Availability}
Computational code and data are available at DOI: 10.5281/zenodo.18348730. All calculations are reproducible using the provided Python implementation.

%------------------------------------------------------------------------------
% REFERENCES
%------------------------------------------------------------------------------
\begin{thebibliography}{99}

\bibitem{lmfdb}
The LMFDB Collaboration, ``The L-functions and Modular Forms Database,'' \url{https://www.lmfdb.org}, 2024.

\bibitem{codata}
E. Tiesinga et al., ``CODATA Recommended Values of the Fundamental Physical Constants: 2022,'' \emph{Rev. Mod. Phys.}, 2024.

\bibitem{e6gut}
F. G\"ursey, P. Ramond, and P. Sikivie, ``A universal gauge theory model based on $E_6$,'' \emph{Phys. Lett. B}, vol. 60, pp. 177--180, 1976.

\bibitem{jordan}
P. Jordan, J. von Neumann, and E. Wigner, ``On an algebraic generalization of the quantum mechanical formalism,'' \emph{Ann. Math.}, vol. 35, pp. 29--64, 1934.

\bibitem{27lines}
A. Cayley, ``On the triple tangent planes of surfaces of the third order,'' \emph{Cambridge and Dublin Math. J.}, vol. 4, pp. 118--132, 1849.

\bibitem{bsd}
B. Birch and H. P. F. Swinnerton-Dyer, ``Notes on elliptic curves. II,'' \emph{J. Reine Angew. Math.}, vol. 218, pp. 79--108, 1965.

\bibitem{string}
M. B. Green, J. H. Schwarz, and E. Witten, \emph{Superstring Theory}, Cambridge University Press, 1987.

\bibitem{mirror}
P. Candelas, X. C. de la Ossa, P. S. Green, and L. Parkes, ``A pair of Calabi-Yau manifolds as an exactly soluble superconformal theory,'' \emph{Nucl. Phys. B}, vol. 359, pp. 21--74, 1991.

\end{thebibliography}

\end{document}
