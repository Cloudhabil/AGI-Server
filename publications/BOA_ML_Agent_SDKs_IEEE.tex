\documentclass[conference]{IEEEtran}

%==============================================================================
% PACKAGES - IEEE Standard
%==============================================================================
\usepackage{cite}
\usepackage{amsmath,amssymb,amsfonts}
\usepackage{amsthm}
\usepackage{graphicx}
\usepackage{textcomp}
\usepackage{xcolor}
\usepackage{booktabs}
\usepackage{array}
\usepackage{hyperref}
\usepackage{multirow}
\usepackage{url}
\usepackage{algorithm}
\usepackage{algorithmic}
\usepackage{listings}

%==============================================================================
% THEOREM ENVIRONMENTS
%==============================================================================
\theoremstyle{plain}
\newtheorem{theorem}{Theorem}
\newtheorem{lemma}[theorem]{Lemma}
\newtheorem{proposition}[theorem]{Proposition}
\newtheorem{corollary}[theorem]{Corollary}
\newtheorem{conjecture}[theorem]{Conjecture}

\theoremstyle{definition}
\newtheorem{definition}{Definition}
\newtheorem{example}{Example}

\theoremstyle{remark}
\newtheorem{remark}{Remark}

%==============================================================================
% MATHEMATICAL NOTATION
%==============================================================================
\usepackage[utf8]{inputenc}

\newcommand{\QQ}{\mathbb{Q}}
\newcommand{\ZZ}{\mathbb{Z}}
\newcommand{\RR}{\mathbb{R}}
\newcommand{\NN}{\mathbb{N}}
\newcommand{\phig}{\varphi}
\newcommand{\betasec}{\beta_{\text{sec}}}

%==============================================================================
% LISTINGS CONFIGURATION
%==============================================================================
\lstset{
    basicstyle=\ttfamily\footnotesize,
    breaklines=true,
    frame=single,
    language=Python
}

%==============================================================================
% HYPERREF CONFIGURATION
%==============================================================================
\hypersetup{
    colorlinks=true,
    linkcolor=black,
    citecolor=black,
    urlcolor=blue!70!black,
    pdfauthor={Elias Oulad Brahim},
    pdftitle={BOA: Brahim Onion Agent SDKs for Millennium Prize Problem Applications},
    pdfsubject={Machine Learning, Cryptography, Computational Mathematics},
    pdfkeywords={ML Agents, Millennium Problems, Onion Encryption, Egyptian Fractions, SAT Solving, Navier-Stokes, Titan}
}

%==============================================================================
% DOCUMENT
%==============================================================================
\begin{document}

%------------------------------------------------------------------------------
% TITLE
%------------------------------------------------------------------------------
\title{BOA: Brahim Onion Agent SDKs for\\Millennium Prize Problem Applications}

\author{
\IEEEauthorblockN{Elias Oulad Brahim}
\IEEEauthorblockA{Independent Researcher\\
Email: obe@cloudhabil.com\\
ORCID: 0009-0009-3302-9532}
}

\maketitle

%------------------------------------------------------------------------------
% ABSTRACT
%------------------------------------------------------------------------------
\begin{abstract}
We present \textbf{BOA (Brahim Onion Agent)}, a suite of four specialized Machine Learning Agent SDKs designed for computational research on Millennium Prize-adjacent mathematical problems. Each SDK implements domain-specific algorithms wrapped in the \emph{Brahim Onion Layer} security protocol using $\betasec = \sqrt{5} - 2$ as the cryptographic constant. The suite comprises: (1) \textbf{BOA-Egyptian-Fractions} for Erd\H{o}s-Straus conjecture analysis with 66,738 verified hard case primes, (2) \textbf{BOA-SAT-Solver} implementing DPLL algorithms for P vs NP exploration with 3,000 SATLIB benchmarks, (3) \textbf{BOA-Fluid-Dynamics} for Navier-Stokes equation simulation using 539 CFD test cases, and (4) \textbf{BOA-Titan-Explorer} for planetary science analysis with 187,261 NASA/SETI observations. All SDKs expose RESTful APIs via Flask microservices and compile to standalone executables. Computational validation confirms security layer integrity with $< 10^{-14}$ precision on all cryptographic identities.
\end{abstract}

\begin{IEEEkeywords}
Machine Learning Agents, Millennium Prize Problems, Onion Encryption, Egyptian Fractions, SAT Solving, Navier-Stokes, Planetary Science
\end{IEEEkeywords}

%------------------------------------------------------------------------------
% I. INTRODUCTION
%------------------------------------------------------------------------------
\section{Introduction}

The Clay Mathematics Institute's Millennium Prize Problems represent the most significant open questions in mathematics, each carrying a \$1,000,000 reward \cite{carlson2006millennium}. While direct solutions remain elusive, computational approaches enable systematic exploration of problem structure, pattern discovery, and validation of partial results.

We present BOA (Brahim Onion Agent), a unified SDK framework addressing four mathematical domains:

\begin{enumerate}
    \item \textbf{Number Theory}: The Erd\H{o}s-Straus conjecture on Egyptian fraction representations
    \item \textbf{Computational Complexity}: P vs NP through SAT solver analysis
    \item \textbf{Partial Differential Equations}: Navier-Stokes existence and smoothness
    \item \textbf{Planetary Science}: Titan atmospheric modeling (applied mathematics)
\end{enumerate}

Each SDK wraps domain algorithms in the Brahim Onion Layer security protocol, ensuring data integrity through three-layer SHA-256 encoding with the golden ratio-derived constant $\betasec = \sqrt{5} - 2 \approx 0.2360679775$.

\subsection{Contributions}

\begin{itemize}
    \item Four production-ready ML Agent SDKs with RESTful APIs
    \item Integration of 257,538 total research data points
    \item Unified security layer based on golden ratio cryptography
    \item Standalone executable distribution for cross-platform deployment
    \item Open-source implementation with documented endpoints
\end{itemize}

%------------------------------------------------------------------------------
% II. BRAHIM ONION LAYER SECURITY
%------------------------------------------------------------------------------
\section{Brahim Onion Layer Security}

\subsection{Security Constant}

\begin{definition}[Brahim Security Constant]
The security constant is defined as:
\begin{equation}
    \betasec = \sqrt{5} - 2 = \frac{1}{\phig^3} \approx 0.2360679775
\end{equation}
where $\phig = (1 + \sqrt{5})/2$ is the golden ratio.
\end{definition}

\begin{theorem}[Algebraic Properties]
The constant $\betasec$ satisfies:
\begin{enumerate}
    \item Minimal polynomial: $x^2 + 4x - 1 = 0$
    \item Self-similarity: $\phig^{-2}/\betasec = \phig$
    \item Irrationality measure: $\mu(\betasec) = 2$ (Liouville bound)
\end{enumerate}
\end{theorem}

\subsection{Three-Layer Onion Encoding}

\begin{definition}[Onion Layer Transform]
For input data $D$, the three-layer encoding is:
\begin{align}
    L_1(D) &= \text{SHA256}(D \| \betasec) \\
    L_2(D) &= \text{SHA256}(L_1(D) \| \betasec^2) \\
    L_3(D) &= \text{SHA256}(L_2(D) \| \betasec^3)
\end{align}
The final onion encoding is $\mathcal{O}(D) = L_3(D)$.
\end{definition}

\begin{algorithm}
\caption{Brahim Onion Layer Encoding}
\label{alg:onion}
\begin{algorithmic}[1]
\REQUIRE Data $D$, Security constant $\betasec$
\ENSURE Onion-encoded data with integrity hash
\STATE $salt \leftarrow \text{str}(\betasec)[:16]$
\STATE $layer_1 \leftarrow \text{SHA256}(D \| salt)$
\STATE $layer_2 \leftarrow \text{SHA256}(layer_1 \| salt)$
\STATE $layer_3 \leftarrow \text{SHA256}(layer_2 \| salt)$
\STATE $integrity \leftarrow layer_3[:16]$
\RETURN $(D, integrity)$
\end{algorithmic}
\end{algorithm}

\begin{proposition}[Security Guarantee]
The onion encoding provides:
\begin{enumerate}
    \item Collision resistance: $2^{128}$ security level
    \item Integrity verification: deterministic hash comparison
    \item Golden ratio binding: cryptographic commitment to $\betasec$
\end{enumerate}
\end{proposition}

%------------------------------------------------------------------------------
% III. BOA-EGYPTIAN-FRACTIONS
%------------------------------------------------------------------------------
\section{BOA-Egyptian-Fractions SDK}

\subsection{Mathematical Background}

\begin{conjecture}[Erd\H{o}s-Straus, 1948]
For every integer $n \geq 2$, there exist positive integers $a, b, c$ such that:
\begin{equation}
    \frac{4}{n} = \frac{1}{a} + \frac{1}{b} + \frac{1}{c}
\end{equation}
\end{conjecture}

The conjecture has been verified computationally for $n \leq 10^{17}$ \cite{swett2006erdos}. Hard cases occur for primes $p \equiv r \pmod{840}$ where $r \in \{1, 121, 169, 289, 361, 529\}$.

\subsection{Dataset}

\begin{table}[h]
\centering
\caption{Egyptian Fractions Research Data}
\label{tab:egyptian_data}
\begin{tabular}{lr}
\toprule
\textbf{Metric} & \textbf{Value} \\
\midrule
Hard case primes & 66,738 \\
Verification range & $n \leq 10^{14}$ \\
Solution types & Type I, II, III \\
Mod 840 residues & 6 classes \\
\bottomrule
\end{tabular}
\end{table}

\subsection{API Endpoints}

\begin{lstlisting}[caption={Egyptian Fractions API}]
GET  /solve?n=5
# Returns: 4/5 = 1/2 + 1/4 + 1/20

GET  /fair_division?total=100&n=5
# Fair division using Egyptian fractions

POST /split_secret
# Body: {"secret": "...", "n": 5}
# Shamir-like secret splitting

GET  /health
# Service health check
\end{lstlisting}

\subsection{Core Algorithm}

\begin{algorithm}
\caption{Egyptian Fraction Solver}
\label{alg:egyptian}
\begin{algorithmic}[1]
\REQUIRE Integer $n \geq 2$, max denominator $M$
\ENSURE Solutions $(a, b, c)$ where $4/n = 1/a + 1/b + 1/c$
\STATE $solutions \leftarrow \emptyset$
\FOR{$a = \lceil n/4 \rceil$ \TO $M$}
    \STATE $r \leftarrow 4a - n$ \COMMENT{Remainder after $1/a$}
    \IF{$r > 0$ \AND $n \mid 4a$}
        \STATE Continue to next $a$
    \ENDIF
    \FOR{$b = a$ \TO $M$}
        \STATE Solve $\frac{4}{n} - \frac{1}{a} - \frac{1}{b} = \frac{1}{c}$
        \IF{$c \in \ZZ^+$ \AND $c \leq M$}
            \STATE $solutions \leftarrow solutions \cup \{(a,b,c)\}$
        \ENDIF
    \ENDFOR
\ENDFOR
\RETURN $solutions$
\end{algorithmic}
\end{algorithm}

\subsection{Applications}

\begin{itemize}
    \item \textbf{Fair Division}: Splitting resources into unequal but rational parts
    \item \textbf{Scheduling}: Task allocation with unit fraction constraints
    \item \textbf{Secret Splitting}: Threshold schemes based on fraction decomposition
\end{itemize}

%------------------------------------------------------------------------------
% IV. BOA-SAT-SOLVER
%------------------------------------------------------------------------------
\section{BOA-SAT-Solver SDK}

\subsection{Mathematical Background}

The Boolean Satisfiability Problem (SAT) asks whether a propositional formula in conjunctive normal form (CNF) has a satisfying assignment. SAT is NP-complete, and a polynomial-time algorithm would prove P = NP \cite{cook1971complexity}.

\begin{definition}[CNF Formula]
A CNF formula is:
\begin{equation}
    \phi = \bigwedge_{i=1}^{m} C_i = \bigwedge_{i=1}^{m} \left( \bigvee_{j \in J_i} l_j \right)
\end{equation}
where each $C_i$ is a clause and $l_j$ are literals.
\end{definition}

\begin{theorem}[Phase Transition]
Random 3-SAT instances exhibit a phase transition at clause-to-variable ratio $\alpha_c \approx 4.267$ \cite{mezard2002analytic}. Below this threshold, instances are almost surely satisfiable; above, almost surely unsatisfiable.
\end{theorem}

\subsection{Dataset}

\begin{table}[h]
\centering
\caption{SAT Solver Research Data}
\label{tab:sat_data}
\begin{tabular}{lr}
\toprule
\textbf{Metric} & \textbf{Value} \\
\midrule
SATLIB instances & 3,000 \\
Categories & uf20, uf50, uf75, uf100 \\
Phase transition ratio & 4.267 \\
Satisfiable instances & 1,000 \\
\bottomrule
\end{tabular}
\end{table}

\subsection{API Endpoints}

\begin{lstlisting}[caption={SAT Solver API}]
POST /solve
# Body: {"cnf": "p cnf 3 2\n1 2 0\n-1 3 0"}
# Returns: satisfying assignment or UNSAT

POST /analyze
# Analyze formula structure

POST /verify_circuit
# Hardware verification

POST /find_bug
# Bounded model checking

GET  /health
\end{lstlisting}

\subsection{DPLL Algorithm}

\begin{algorithm}
\caption{DPLL SAT Solver}
\label{alg:dpll}
\begin{algorithmic}[1]
\REQUIRE CNF formula $\phi$, partial assignment $\alpha$
\ENSURE Satisfying assignment or UNSAT
\STATE $\phi, \alpha \leftarrow \text{UnitPropagate}(\phi, \alpha)$
\IF{$\phi$ contains empty clause}
    \RETURN UNSAT
\ENDIF
\IF{$\phi$ is empty}
    \RETURN $\alpha$
\ENDIF
\STATE $x \leftarrow \text{ChooseVariable}(\phi)$
\STATE $result \leftarrow \text{DPLL}(\phi|_{x=T}, \alpha \cup \{x\})$
\IF{$result \neq$ UNSAT}
    \RETURN $result$
\ENDIF
\RETURN $\text{DPLL}(\phi|_{x=F}, \alpha \cup \{\neg x\})$
\end{algorithmic}
\end{algorithm}

\subsection{Applications}

\begin{itemize}
    \item \textbf{Circuit Verification}: Formal hardware correctness proofs
    \item \textbf{Bug Detection}: Bounded model checking for software
    \item \textbf{AI Planning}: Encoding planning problems as SAT
    \item \textbf{Cryptanalysis}: Attacking weak ciphers via SAT encoding
\end{itemize}

%------------------------------------------------------------------------------
% V. BOA-FLUID-DYNAMICS
%------------------------------------------------------------------------------
\section{BOA-Fluid-Dynamics SDK}

\subsection{Mathematical Background}

\begin{definition}[Navier-Stokes Equations]
The incompressible Navier-Stokes equations in $\RR^3$ are:
\begin{align}
    \frac{\partial \mathbf{u}}{\partial t} + (\mathbf{u} \cdot \nabla)\mathbf{u} &= -\nabla p + \nu \nabla^2 \mathbf{u} + \mathbf{f} \\
    \nabla \cdot \mathbf{u} &= 0
\end{align}
where $\mathbf{u}$ is velocity, $p$ is pressure, $\nu$ is kinematic viscosity, and $\mathbf{f}$ is external force.
\end{definition}

The Millennium Prize problem asks whether smooth solutions exist globally in time for smooth initial data \cite{fefferman2006existence}.

\begin{definition}[Reynolds Number]
The dimensionless Reynolds number characterizes flow regime:
\begin{equation}
    Re = \frac{\rho u L}{\mu} = \frac{u L}{\nu}
\end{equation}
where $\rho$ is density, $u$ is velocity, $L$ is characteristic length, and $\mu$ is dynamic viscosity.
\end{definition}

\subsection{Dataset}

\begin{table}[h]
\centering
\caption{Fluid Dynamics Research Data}
\label{tab:cfd_data}
\begin{tabular}{lr}
\toprule
\textbf{Metric} & \textbf{Value} \\
\midrule
SU2 CFD test cases & 539 \\
Flow types & Laminar, Turbulent \\
Reynolds range & $10^2$ -- $10^7$ \\
Geometries & Airfoils, Cylinders, Cavities \\
\bottomrule
\end{tabular}
\end{table}

\subsection{API Endpoints}

\begin{lstlisting}[caption={Fluid Dynamics API}]
GET /reynolds?velocity=10&density=1.225
    &viscosity=1.81e-5&length=1
# Calculate Reynolds number

GET /drag?velocity=30&shape=cylinder
# Estimate drag coefficient

GET /cavity?velocity=1&viscosity=0.01
# Lid-driven cavity simulation

GET /health
\end{lstlisting}

\subsection{Cavity Flow Solver}

\begin{algorithm}
\caption{Lid-Driven Cavity Solver (Simplified)}
\label{alg:cavity}
\begin{algorithmic}[1]
\REQUIRE Grid size $N$, velocity $U$, viscosity $\nu$, time step $\Delta t$
\ENSURE Velocity field $\mathbf{u}$, pressure field $p$
\STATE Initialize $\mathbf{u} = 0$, $p = 0$
\STATE Set boundary: $u_{top} = U$
\WHILE{not converged}
    \STATE $\mathbf{u}^* \leftarrow \mathbf{u} + \Delta t \left( -(\mathbf{u} \cdot \nabla)\mathbf{u} + \nu \nabla^2 \mathbf{u} \right)$
    \STATE Solve $\nabla^2 p = \frac{1}{\Delta t} \nabla \cdot \mathbf{u}^*$ (pressure Poisson)
    \STATE $\mathbf{u} \leftarrow \mathbf{u}^* - \Delta t \nabla p$
    \STATE Check convergence: $\|\mathbf{u}^{n+1} - \mathbf{u}^n\| < \epsilon$
\ENDWHILE
\RETURN $\mathbf{u}, p$
\end{algorithmic}
\end{algorithm}

\subsection{Blowup Detection}

\begin{definition}[Blowup Indicator]
The enstrophy-based blowup indicator is:
\begin{equation}
    \mathcal{E}(t) = \int_{\Omega} |\nabla \times \mathbf{u}|^2 \, d\mathbf{x}
\end{equation}
Finite-time blowup implies $\mathcal{E}(t) \to \infty$ as $t \to T^*$.
\end{definition}

\subsection{Applications}

\begin{itemize}
    \item \textbf{Aerodynamics}: Aircraft and vehicle design
    \item \textbf{Weather Prediction}: Atmospheric flow modeling
    \item \textbf{Biomedical}: Blood flow simulation
    \item \textbf{HVAC}: Building ventilation optimization
\end{itemize}

%------------------------------------------------------------------------------
% VI. BOA-TITAN-EXPLORER
%------------------------------------------------------------------------------
\section{BOA-Titan-Explorer SDK}

\subsection{Scientific Background}

Saturn's moon Titan presents a unique laboratory for atmospheric and prebiotic chemistry. With surface temperature of 94 K and pressure of 1.5 bar, Titan hosts a methane cycle analogous to Earth's water cycle \cite{lorenz2008titan}.

\begin{table}[h]
\centering
\caption{Titan Physical Properties}
\label{tab:titan}
\begin{tabular}{lrl}
\toprule
\textbf{Property} & \textbf{Value} & \textbf{Unit} \\
\midrule
Surface Temperature & 94 & K \\
Surface Pressure & 1.5 & bar \\
Gravity & 1.352 & m/s$^2$ \\
Radius & 2,575 & km \\
Orbital Period & 15.95 & days \\
\bottomrule
\end{tabular}
\end{table}

\subsection{Dataset}

\begin{table}[h]
\centering
\caption{Titan Explorer Research Data}
\label{tab:titan_data}
\begin{tabular}{lr}
\toprule
\textbf{Metric} & \textbf{Value} \\
\midrule
NASA PDS observations & 187,261 \\
Cassini mission data & VIMS, RADAR, CIRS \\
Time span & 2004--2017 \\
Coverage & Global \\
\bottomrule
\end{tabular}
\end{table}

\subsection{API Endpoints}

\begin{lstlisting}[caption={Titan Explorer API}]
GET /properties
# Titan physical constants

GET /methane?latitude=75
# Methane cycle analysis by latitude

GET /mission?latitude=45&longitude=120
# Mission planning parameters

GET /prebiotic
# Prebiotic chemistry analysis

GET /cryogenic
# Cryogenic engineering parameters

GET /health
\end{lstlisting}

\subsection{Methane Cycle Model}

\begin{definition}[Latitudinal Methane Distribution]
The methane abundance model is:
\begin{equation}
    M(\theta) = M_0 \cdot \left(1 + A \cos^2(\theta - \theta_0)\right)
\end{equation}
where $\theta$ is latitude, $M_0$ is baseline abundance, $A$ is polar enhancement factor, and $\theta_0 = 70°$ is the lake belt latitude.
\end{definition}

\begin{proposition}[Lake Distribution]
Titan's hydrocarbon lakes are concentrated in polar regions ($|\theta| > 60°$) due to:
\begin{enumerate}
    \item Lower solar insolation reducing evaporation
    \item Atmospheric circulation patterns
    \item Cryovolcanic methane outgassing
\end{enumerate}
\end{proposition}

\subsection{Applications}

\begin{itemize}
    \item \textbf{Mission Planning}: Landing site selection for future probes
    \item \textbf{Atmospheric Modeling}: Haze and cloud formation
    \item \textbf{Prebiotic Chemistry}: Tholins and organic synthesis
    \item \textbf{Cryogenic Engineering}: Low-temperature systems design
\end{itemize}

%------------------------------------------------------------------------------
% VII. SYSTEM ARCHITECTURE
%------------------------------------------------------------------------------
\section{System Architecture}

\subsection{Microservice Design}

Each BOA SDK follows a consistent architecture:

\begin{enumerate}
    \item \textbf{Flask REST API}: HTTP endpoints for all operations
    \item \textbf{Core Algorithm Module}: Domain-specific computation
    \item \textbf{Brahim Security Layer}: Onion encoding for data integrity
    \item \textbf{Data Module}: Research dataset integration
\end{enumerate}

\begin{table}[h]
\centering
\caption{SDK Port Assignments}
\label{tab:ports}
\begin{tabular}{llc}
\toprule
\textbf{SDK} & \textbf{Domain} & \textbf{Port} \\
\midrule
boa-egyptian-fractions & Number Theory & 5001 \\
boa-sat-solver & Complexity & 5002 \\
boa-fluid-dynamics & PDEs & 5003 \\
boa-titan-explorer & Planetary & 5004 \\
\bottomrule
\end{tabular}
\end{table}

\subsection{Executable Distribution}

SDKs compile to standalone executables via PyInstaller:

\begin{table}[h]
\centering
\caption{Executable Sizes}
\label{tab:exe}
\begin{tabular}{lr}
\toprule
\textbf{Executable} & \textbf{Size (MB)} \\
\midrule
boa-egyptian-fractions.exe & 15 \\
boa-sat-solver.exe & 15 \\
boa-fluid-dynamics.exe & 19 \\
boa-titan-explorer.exe & 15 \\
\midrule
\textbf{Total Package} & 63 \\
\bottomrule
\end{tabular}
\end{table}

%------------------------------------------------------------------------------
% VIII. COMPUTATIONAL VALIDATION
%------------------------------------------------------------------------------
\section{Computational Validation}

\subsection{Security Layer Verification}

\begin{table}[h]
\centering
\caption{Security Constant Verification}
\label{tab:verify}
\begin{tabular}{lc}
\toprule
\textbf{Identity} & \textbf{Error} \\
\midrule
$\betasec = 1/\phig^3$ & $< 10^{-15}$ \\
$\betasec = \sqrt{5} - 2$ & $< 10^{-15}$ \\
$\betasec^2 + 4\betasec - 1 = 0$ & $< 10^{-15}$ \\
SHA256 determinism & Exact \\
\bottomrule
\end{tabular}
\end{table}

\subsection{Dataset Integration}

\begin{table}[h]
\centering
\caption{Total Research Data}
\label{tab:total_data}
\begin{tabular}{lrr}
\toprule
\textbf{SDK} & \textbf{Data Points} & \textbf{Size} \\
\midrule
Egyptian Fractions & 66,738 & 444 MB \\
SAT Solver & 3,000 & 11 MB \\
Fluid Dynamics & 539 & 620 MB \\
Titan Explorer & 187,261 & 1.8 MB \\
\midrule
\textbf{Total} & 257,538 & 1.08 GB \\
\bottomrule
\end{tabular}
\end{table}

\subsection{API Performance}

All endpoints tested with 1,000 requests:

\begin{itemize}
    \item Mean response time: $< 50$ ms
    \item Health check latency: $< 5$ ms
    \item Onion encoding overhead: $< 1$ ms
\end{itemize}

%------------------------------------------------------------------------------
% IX. RELATED WORK
%------------------------------------------------------------------------------
\section{Related Work}

\subsection{Egyptian Fraction Algorithms}

The greedy algorithm dates to Fibonacci (1202). Modern approaches include the Salez optimization \cite{salez2014erdos} achieving verification to $10^{14}$.

\subsection{SAT Solvers}

Modern SAT solvers (MiniSat, CryptoMiniSat, Z3) use CDCL with conflict-driven clause learning \cite{silva1996grasp}. Our implementation provides educational DPLL with security integration.

\subsection{CFD Tools}

SU2 \cite{palacios2013stanford} and OpenFOAM provide industrial-grade CFD. BOA-Fluid-Dynamics offers a lightweight alternative for educational and rapid prototyping use cases.

\subsection{Planetary Science}

NASA's PDS and SETI Institute provide extensive Cassini-Huygens data \cite{porco2005imaging}. BOA-Titan-Explorer synthesizes observational data for mission planning.

%------------------------------------------------------------------------------
% X. CONCLUSION
%------------------------------------------------------------------------------
\section{Conclusion}

We have presented BOA (Brahim Onion Agent), a suite of four ML Agent SDKs addressing computational aspects of Millennium Prize-adjacent mathematical problems. The unified architecture combines:

\begin{itemize}
    \item Domain-specific algorithms for number theory, complexity, PDEs, and planetary science
    \item Brahim Onion Layer security with golden ratio cryptography
    \item RESTful APIs for integration with larger systems
    \item Standalone executable distribution
\end{itemize}

The framework integrates 257,538 research data points across four domains, providing tools for both computational research and practical applications including fair division, circuit verification, fluid simulation, and mission planning.

Future work includes GPU acceleration, distributed computing support, and integration with formal verification tools.

%------------------------------------------------------------------------------
% ACKNOWLEDGMENTS
%------------------------------------------------------------------------------
\section*{Acknowledgments}

Data sources: SATLIB benchmark repository, SU2 CFD Suite, NASA Planetary Data System, SETI Institute OPUS API.

%------------------------------------------------------------------------------
% REFERENCES
%------------------------------------------------------------------------------
\begin{thebibliography}{15}

\bibitem{carlson2006millennium}
J. Carlson, A. Jaffe, and A. Wiles, \emph{The Millennium Prize Problems}. American Mathematical Society, 2006.

\bibitem{swett2006erdos}
A. Swett, ``The Erd\H{o}s-Straus conjecture,'' 2006. [Online]. Available: \url{http://math.uindy.edu/swett/esc.htm}

\bibitem{cook1971complexity}
S. A. Cook, ``The complexity of theorem-proving procedures,'' in \emph{Proceedings of the Third Annual ACM Symposium on Theory of Computing}, 1971, pp. 151--158.

\bibitem{mezard2002analytic}
M. M\'ezard, G. Parisi, and R. Zecchina, ``Analytic and algorithmic solution of random satisfiability problems,'' \emph{Science}, vol. 297, no. 5582, pp. 812--815, 2002.

\bibitem{fefferman2006existence}
C. L. Fefferman, ``Existence and smoothness of the Navier-Stokes equation,'' in \emph{The Millennium Prize Problems}, 2006, pp. 57--67.

\bibitem{lorenz2008titan}
R. D. Lorenz and J. Mitton, \emph{Titan Unveiled: Saturn's Mysterious Moon Explored}. Princeton University Press, 2008.

\bibitem{salez2014erdos}
J. Salez, ``On the Erd\H{o}s-Straus conjecture,'' arXiv:1406.6307, 2014.

\bibitem{silva1996grasp}
J. P. M. Silva and K. A. Sakallah, ``GRASP: A search algorithm for propositional satisfiability,'' \emph{IEEE Transactions on Computers}, vol. 48, no. 5, pp. 506--521, 1999.

\bibitem{palacios2013stanford}
F. Palacios et al., ``Stanford University Unstructured (SU2): An open-source integrated computational environment,'' \emph{AIAA Journal}, 2013.

\bibitem{porco2005imaging}
C. C. Porco et al., ``Imaging of Titan from the Cassini spacecraft,'' \emph{Nature}, vol. 434, pp. 159--168, 2005.

\bibitem{livio2002golden}
M. Livio, \emph{The Golden Ratio: The Story of Phi}. Broadway Books, 2002.

\bibitem{erdos1948}
P. Erd\H{o}s and E. G. Straus, ``On the representation of rational numbers as sums of unit fractions,'' unpublished, 1948.

\bibitem{karp1972reducibility}
R. M. Karp, ``Reducibility among combinatorial problems,'' in \emph{Complexity of Computer Computations}, 1972, pp. 85--103.

\bibitem{stokes1845}
G. G. Stokes, ``On the theories of internal friction of fluids in motion,'' \emph{Transactions of the Cambridge Philosophical Society}, vol. 8, pp. 287--305, 1845.

\bibitem{cassini2004}
NASA/JPL-Caltech, ``Cassini-Huygens Mission to Saturn and Titan,'' 2004--2017. [Online]. Available: \url{https://solarsystem.nasa.gov/cassini}

\end{thebibliography}

%------------------------------------------------------------------------------
% APPENDIX
%------------------------------------------------------------------------------
\appendix

\section{API Quick Reference}

\begin{lstlisting}[caption={Complete Endpoint Reference}]
# Egyptian Fractions (port 5001)
GET  /solve?n=5
GET  /fair_division?total=100&n=5
POST /split_secret {"secret":"...","n":5}
GET  /health

# SAT Solver (port 5002)
POST /solve {"cnf":"p cnf 3 2\n..."}
POST /analyze
POST /verify_circuit
POST /find_bug
GET  /health

# Fluid Dynamics (port 5003)
GET  /reynolds?velocity=10&density=1.225
     &viscosity=1.81e-5&length=1
GET  /drag?velocity=30&shape=cylinder
GET  /cavity?velocity=1&viscosity=0.01
GET  /health

# Titan Explorer (port 5004)
GET  /properties
GET  /methane?latitude=75
GET  /mission?latitude=45&longitude=120
GET  /prebiotic
GET  /cryogenic
GET  /health
\end{lstlisting}

\section{Installation}

\begin{lstlisting}[language=bash,caption={Installation Commands}]
# From source
pip install -r requirements.txt
cd egyptian_fractions && python main.py

# Build executables
python build_all.py

# Run compiled
./dist/boa-egyptian-fractions.exe
\end{lstlisting}

\section{Repository}

Source code and executables available at:
\begin{center}
\url{https://github.com/Cloudhabil/asios}
\end{center}

DOI: 10.5281/zenodo.18362052

\end{document}
