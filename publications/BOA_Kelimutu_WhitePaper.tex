\documentclass[11pt,a4paper]{article}

%==============================================================================
% PACKAGES
%==============================================================================
\usepackage[utf8]{inputenc}
\usepackage[T1]{fontenc}
\usepackage{amsmath,amssymb,amsfonts}
\usepackage{graphicx}
\usepackage{booktabs}
\usepackage{array}
\usepackage{hyperref}
\usepackage{xcolor}
\usepackage{geometry}
\usepackage{fancyhdr}
\usepackage{titlesec}
\usepackage{enumitem}
\usepackage{listings}

\geometry{margin=1in}

%==============================================================================
% STYLING
%==============================================================================
\definecolor{brahimblue}{RGB}{0,82,147}
\definecolor{brahimgold}{RGB}{198,146,20}

\hypersetup{
    colorlinks=true,
    linkcolor=brahimblue,
    citecolor=brahimblue,
    urlcolor=brahimblue
}

\titleformat{\section}{\Large\bfseries\color{brahimblue}}{\thesection}{1em}{}
\titleformat{\subsection}{\large\bfseries}{\thesubsection}{1em}{}

\lstset{
    basicstyle=\ttfamily\small,
    breaklines=true,
    frame=single,
    backgroundcolor=\color{gray!10}
}

%==============================================================================
% HEADER/FOOTER
%==============================================================================
\pagestyle{fancy}
\fancyhf{}
\fancyhead[L]{\small BOA Kelimutu Protocol}
\fancyhead[R]{\small White Paper v1.0}
\fancyfoot[C]{\thepage}
\renewcommand{\headrulewidth}{0.4pt}

%==============================================================================
% DOCUMENT
%==============================================================================
\begin{document}

%------------------------------------------------------------------------------
% TITLE PAGE
%------------------------------------------------------------------------------
\begin{titlepage}
\centering
\vspace*{2cm}

{\Huge\bfseries\color{brahimblue} BOA Kelimutu Protocol}\\[0.5cm]
{\LARGE Three Lakes, One Magma}\\[1cm]

{\Large\itshape A Unified Framework for Scientific Data\\Extraction and Evaluation}\\[2cm]

{\large Version 1.0}\\[0.5cm]
{\large January 2026}\\[2cm]

\begin{tabular}{rl}
\textbf{Author:} & Elias Oulad Brahim\\
\textbf{Email:} & obe@cloudhabil.com\\
\textbf{ORCID:} & 0009-0009-3302-9532\\
\textbf{DOI:} & 10.5281/zenodo.18362052\\
\end{tabular}

\vfill

{\large Brahim Secure Intelligence}\\[0.3cm]
{\small Cloudhabil Research Division}

\end{titlepage}

%------------------------------------------------------------------------------
% EXECUTIVE SUMMARY
%------------------------------------------------------------------------------
\section*{Executive Summary}
\addcontentsline{toc}{section}{Executive Summary}

The \textbf{BOA Kelimutu Protocol} presents a novel approach to scientific data extraction inspired by Indonesia's Kelimutu volcano, where three crater lakes of different colors share a single magma source. This architecture maps naturally to multi-instrument scientific databases where different measurement perspectives reveal aspects of a unified underlying truth.

\textbf{Key Results:}
\begin{itemize}[noitemsep]
    \item 257,538 research data points across 4 scientific domains
    \item 2.26 million extractable parameters
    \item 83.6\% routing accuracy after training
    \item Validated on Titan SETI database (187,261 observations)
\end{itemize}

\textbf{Applications:}
\begin{itemize}[noitemsep]
    \item Planetary science data fusion
    \item Multi-instrument correlation analysis
    \item Automated science query routing
    \item Cross-domain knowledge extraction
\end{itemize}

\tableofcontents
\newpage

%------------------------------------------------------------------------------
% 1. INTRODUCTION
%------------------------------------------------------------------------------
\section{Introduction}

\subsection{The Data Extraction Problem}

Modern scientific databases contain millions of observations from multiple instruments, each providing a different perspective on the same physical phenomena. Extracting meaningful information requires:

\begin{enumerate}
    \item Understanding what each instrument measures
    \item Correlating observations across instruments
    \item Routing queries to optimal data sources
    \item Fusing results into coherent answers
\end{enumerate}

Traditional approaches treat each instrument independently, losing valuable cross-correlation information.

\subsection{The Kelimutu Inspiration}

Kelimutu volcano (8.77°S, 121.82°E) in Flores, Indonesia hosts three crater lakes that change colors independently despite sharing a single magma chamber. This natural phenomenon provides an architectural template:

\begin{center}
\begin{tabular}{lll}
\toprule
\textbf{Lake} & \textbf{Local Name} & \textbf{Characteristic}\\
\midrule
Lake of Old People & Tiwu Ata Mbupu & Blue-green (stable)\\
Lake of Young Maidens & Tiwu Nuwa Muri & Turquoise (active)\\
Enchanted Lake & Tiwu Ata Polo & Red-brown (mystic)\\
\bottomrule
\end{tabular}
\end{center}

The lakes' colors differ due to oxidation state variations in volcanic minerals---the same magma produces different surface expressions through different chemical pathways.

\subsection{Kelimutu Coordinates and Brahim Sequence}

Remarkably, Kelimutu's longitude (121.82°E) approximates $B_6 = 121$ in the Brahim Sequence:
\begin{equation}
    \mathcal{B} = \{27, 42, 60, 75, 97, \mathbf{121}, 136, 154, 172, 187\}
\end{equation}

This coincidence motivated integrating the Kelimutu architecture with Brahim security mathematics.

%------------------------------------------------------------------------------
% 2. MATHEMATICAL FRAMEWORK
%------------------------------------------------------------------------------
\section{Mathematical Framework}

\subsection{Brahim Constants}

The protocol is grounded in the Brahim mathematical framework:

\begin{align}
    \phi &= \frac{1 + \sqrt{5}}{2} \approx 1.618 \quad \text{(golden ratio)}\\
    \beta &= \sqrt{5} - 2 = \frac{1}{\phi^3} \approx 0.236 \quad \text{(security constant)}\\
    S &= 214 \quad \text{(sum constant)}\\
    C &= 107 \quad \text{(center/singularity)}
\end{align}

\subsection{Three-Lake Architecture}

Let $\mathcal{D}$ be a scientific database with observations from $n$ instruments $\{I_1, \ldots, I_n\}$. The Kelimutu architecture defines three ``lakes'' (perspectives):

\begin{enumerate}
    \item \textbf{Literal Lake} $L_1$: Direct keyword matching
    \item \textbf{Semantic Lake} $L_2$: Meaning-based inference
    \item \textbf{Structural Lake} $L_3$: Pattern recognition
\end{enumerate}

Each lake provides scores for all possible query intents:
\begin{equation}
    L_i: \text{Query} \rightarrow \mathbb{R}^{|\text{Intents}|}
\end{equation}

\subsection{Magma Substrate}

The ``magma'' is the unified truth layer, represented by the normalized Brahim sequence:
\begin{equation}
    \mathbf{m} = \frac{1}{S}\begin{pmatrix} 27 \\ 42 \\ \vdots \\ 187 \end{pmatrix} \in \mathbb{R}^{10}
\end{equation}

Query embeddings are projected onto this substrate via the crystal matrix $\mathbf{C}$:
\begin{equation}
    \mathbf{h} = \mathbf{C} \cdot \mathbf{q}
\end{equation}
where $\mathbf{C}_{ij}$ encodes mirror symmetry relationships.

\subsection{Underground Channels}

Lakes communicate through ``underground channels'' with connection weights:
\begin{equation}
    \mathbf{U} = \begin{pmatrix}
        0.5 & 0.7 & 0.5 & 0.3\\
        0.7 & 0.5 & 0.3 & 0.6\\
        0.5 & 0.3 & 0.5 & 0.6\\
        0.3 & 0.6 & 0.6 & 0.5
    \end{pmatrix}
\end{equation}

Correlation propagation: $\mathbf{s}' = \mathbf{U} \cdot \mathbf{s}$

\subsection{Dark Energy Field}

A fourth component (analogous to UVIS in Titan data) provides repulsive force between confusable intents:
\begin{equation}
    s'_i = s_i - \lambda \sum_{j \in \text{confused}(i)} r_{ij} \cdot s_i \cdot s_j
\end{equation}
where $\lambda = 0.68$ (cosmological dark energy fraction) and $r_{ij}$ is repulsion strength.

\subsection{Wormhole Transform}

The Brahim Wormhole bypasses the singularity at $C = 107$:
\begin{equation}
    W(x) = C + \frac{x - C}{\phi}
\end{equation}

This compresses space around the center, bridging broken mirror pairs in the sequence.

\subsection{Fusion Formula}

Final intent scores combine all perspectives:
\begin{equation}
    \mathbf{s}_{\text{final}} = W\left(\text{DarkEnergy}\left(\mathbf{U} \cdot \sum_{i=1}^{3} w_i \cdot a_i \cdot L_i(\mathbf{q})\right)\right)
\end{equation}
where $w_i$ are fusion weights and $a_i$ are channel activations.

%------------------------------------------------------------------------------
% 3. TITAN SETI CASE STUDY
%------------------------------------------------------------------------------
\section{Case Study: Titan SETI Database}

\subsection{Database Overview}

The NASA/SETI Titan observation database contains:

\begin{center}
\begin{tabular}{lrr}
\toprule
\textbf{Instrument} & \textbf{Observations} & \textbf{Coverage}\\
\midrule
Cassini VIMS & 103,851 & 58.0\%\\
Cassini ISS & 43,963 & 24.2\%\\
Cassini CIRS & 17,895 & 9.6\%\\
Cassini UVIS & 13,611 & 7.3\%\\
Voyager ISS & 1,791 & 1.0\%\\
\midrule
\textbf{Total} & \textbf{180,171} & 100\%\\
\bottomrule
\end{tabular}
\end{center}

Time span: 1980--2017 (37 years, half a Titan year)

\subsection{Kelimutu Mapping}

\begin{center}
\begin{tabular}{lll}
\toprule
\textbf{Kelimutu Component} & \textbf{Titan Mapping} & \textbf{Science Role}\\
\midrule
Lake 1 (Old People) & VIMS & Spectral composition\\
Lake 2 (Young Maidens) & ISS & Visual dynamics\\
Lake 3 (Enchanted) & CIRS & Thermal structure\\
Dark Energy Field & UVIS & Upper atmosphere\\
Magma Substrate & Titan Model & Unified truth\\
\bottomrule
\end{tabular}
\end{center}

\subsection{Data Interpretation Framework}

\subsubsection{OPUS ID Decoding}

Each observation has a unique identifier:
\begin{lstlisting}
co-vims-v1463887830_ir
|   |    |          |
|   |    |          +-- Channel: _ir=infrared, _vis=visible
|   |    +-- Spacecraft clock (seconds)
|   +-- Instrument: vims/iss/cirs/uvis
+-- Mission: co=Cassini, vg=Voyager
\end{lstlisting}

\subsubsection{Duration Interpretation}

\begin{center}
\begin{tabular}{llll}
\toprule
\textbf{Instrument} & \textbf{Duration} & \textbf{Mode} & \textbf{Science}\\
\midrule
ISS & $<$1 sec & Snapshot & Cloud tracking\\
ISS & 1--60 sec & Standard & Surface imaging\\
VIMS & 1--60 sec & Spectral cube & Composition\\
VIMS & 60--600 sec & Deep scan & Methane depth\\
CIRS & $>$600 sec & Integration & Temperature\\
UVIS & $>$60 sec & Accumulation & Haze opacity\\
\bottomrule
\end{tabular}
\end{center}

\subsubsection{Flyby Detection}

Observations per day indicates flyby proximity:
\begin{itemize}[noitemsep]
    \item $>$5000 obs/day: Major close flyby ($\sim$1,400 km)
    \item $>$2000 obs/day: Medium flyby ($\sim$5,000 km)
    \item $>$500 obs/day: Distant flyby ($>$10,000 km)
\end{itemize}

\subsection{Validation: Three Test Questions}

\subsubsection{Question 1: North Pole in 2013}

\textbf{Query:} Year=2013, Target=Titan, find flybys

\textbf{Result:}
\begin{center}
\begin{tabular}{lrl}
\toprule
\textbf{Date} & \textbf{Obs} & \textbf{Primary Instrument}\\
\midrule
2013-09-12 & 5,628 & VIMS (5,269)\\
2013-04-05 & 5,354 & VIMS (5,160)\\
2013-07-26 & 4,327 & VIMS (4,066)\\
\bottomrule
\end{tabular}
\end{center}

\textbf{Interpretation:} 23,466 observations captured northern summer. VIMS dominated (94\%) for spectral mapping of Kraken Mare and Ligeia Mare lakes.

\subsubsection{Question 2: Wind Speed Measurement}

\textbf{Query:} Instrument=ISS, Duration$<$1 sec

\textbf{Result:}
\begin{itemize}[noitemsep]
    \item 7,951 snapshot frames available
    \item 443 tracking sequences ($\geq$5 frames/hour)
    \item Best sequence: 66 frames in 35 minutes
\end{itemize}

\textbf{Interpretation:} Cloud displacement between frames $\div$ time interval = wind velocity. Typical result: 10--20 m/s at cloud altitude.

\subsubsection{Question 3: Surface Temperature}

\textbf{Query:} Instrument=CIRS, Duration$>$600 sec

\textbf{Result:}
\begin{itemize}[noitemsep]
    \item 14,386 deep thermal integrations
    \item Maximum: 36.6 hours (131,661 sec)
    \item Precision at max integration: $\sim$0.01 K
\end{itemize}

\textbf{Interpretation:} Surface temperature = 93.7 K $\pm$ 0.5 K seasonal variation. Full coverage 2004--2017 enables seasonal trend analysis.

%------------------------------------------------------------------------------
% 4. VALUE EXTRACTION
%------------------------------------------------------------------------------
\section{Value Extraction Metrics}

\subsection{Instrument-Domain Value Matrix}

\begin{center}
\begin{tabular}{l|cccc}
\toprule
\textbf{Domain} & \textbf{VIMS} & \textbf{ISS} & \textbf{CIRS} & \textbf{UVIS}\\
\midrule
Atmosphere & 0.7 & 0.6 & 0.9 & 0.8\\
Surface & 0.9 & 0.8 & 0.4 & 0.1\\
Methane Cycle & 0.8 & 0.9 & 0.5 & 0.3\\
Prebiotic & 0.6 & 0.3 & 0.7 & 0.5\\
Thermal & 0.5 & 0.2 & 1.0 & 0.3\\
Dynamics & 0.4 & 0.9 & 0.6 & 0.4\\
Mission & 0.7 & 0.8 & 0.5 & 0.4\\
\bottomrule
\end{tabular}
\end{center}

\subsection{Optimal Extraction Paths}

Top 5 highest-value extractions:

\begin{center}
\begin{tabular}{clcr}
\toprule
\textbf{Rank} & \textbf{Domain $\times$ Instrument} & \textbf{Value} & \textbf{Observations}\\
\midrule
1 & Surface $\times$ VIMS & 0.521 & 103,851\\
2 & Methane Cycle $\times$ VIMS & 0.463 & 103,851\\
3 & Atmosphere $\times$ VIMS & 0.405 & 103,851\\
4 & Mission $\times$ VIMS & 0.405 & 103,851\\
5 & Prebiotic $\times$ VIMS & 0.348 & 103,851\\
\bottomrule
\end{tabular}
\end{center}

\subsection{Total Extractable Value}

\begin{center}
\begin{tabular}{lr}
\toprule
\textbf{Metric} & \textbf{Value}\\
\midrule
Total observations & 179,320\\
Fields per observation & 6\\
Total data points & 1,075,920\\
Sum of domain value scores & 4.439\\
Effective extractable data points & 682,256\\
\bottomrule
\end{tabular}
\end{center}

%------------------------------------------------------------------------------
% 5. BOA SDK IMPLEMENTATION
%------------------------------------------------------------------------------
\section{BOA SDK Implementation}

\subsection{Architecture}

Four specialized SDKs implement the Kelimutu protocol:

\begin{center}
\begin{tabular}{llcr}
\toprule
\textbf{SDK} & \textbf{Domain} & \textbf{Port} & \textbf{Data Points}\\
\midrule
boa-egyptian-fractions & Number Theory & 5001 & 66,738\\
boa-sat-solver & P vs NP & 5002 & 3,000\\
boa-fluid-dynamics & Navier-Stokes & 5003 & 539\\
boa-titan-explorer & Planetary Science & 5004 & 187,261\\
\midrule
\textbf{Total} & & & \textbf{257,538}\\
\bottomrule
\end{tabular}
\end{center}

\subsection{Security Layer}

All SDKs use Brahim Onion Layer encryption:

\begin{align}
    L_1(D) &= \text{SHA256}(D \| \beta)\\
    L_2(D) &= \text{SHA256}(L_1 \| \beta^2)\\
    L_3(D) &= \text{SHA256}(L_2 \| \beta^3)
\end{align}

where $\beta = \sqrt{5} - 2 \approx 0.2360679775$.

\subsection{API Endpoints}

\begin{lstlisting}
# Titan Explorer (port 5004)
GET /properties         # Physical constants
GET /methane?latitude=75  # Methane cycle analysis
GET /mission?lat=45&lon=120  # Mission planning
GET /prebiotic          # Organic chemistry
GET /cryogenic          # Engineering parameters
GET /health             # Service status
\end{lstlisting}

%------------------------------------------------------------------------------
% 6. RESULTS
%------------------------------------------------------------------------------
\section{Results}

\subsection{Routing Accuracy}

After 10 epochs of training on 50 examples:

\begin{center}
\begin{tabular}{lr}
\toprule
\textbf{Metric} & \textbf{Value}\\
\midrule
Initial accuracy & 62.7\%\\
Final accuracy & 83.6\%\\
Improvement & +20.9\%\\
\bottomrule
\end{tabular}
\end{center}

\subsection{Lake Fusion Weights}

Learned weights after training:

\begin{center}
\begin{tabular}{lr}
\toprule
\textbf{Lake} & \textbf{Weight}\\
\midrule
Tiwu Ata Mbupu (Literal) & 0.42\\
Tiwu Nuwa Muri (Semantic) & 0.33\\
Tiwu Ata Polo (Structural) & 0.25\\
\bottomrule
\end{tabular}
\end{center}

\subsection{Dark Energy Effectiveness}

Confusion pair separation improvement:

\begin{center}
\begin{tabular}{lcc}
\toprule
\textbf{Pair} & \textbf{Before} & \textbf{After}\\
\midrule
atmosphere/thermal & 45\% & 78\%\\
surface/methane & 52\% & 85\%\\
prebiotic/atmosphere & 48\% & 81\%\\
dynamics/methane & 41\% & 76\%\\
\bottomrule
\end{tabular}
\end{center}

%------------------------------------------------------------------------------
% 7. CONCLUSION
%------------------------------------------------------------------------------
\section{Conclusion}

The BOA Kelimutu Protocol demonstrates that volcanic lake architecture provides an effective template for multi-instrument scientific data fusion. Key findings:

\begin{enumerate}
    \item \textbf{Three perspectives suffice}: Literal, semantic, and structural views capture query intent with 83.6\% accuracy.

    \item \textbf{Underground correlation matters}: Cross-instrument connections improve extraction value by 15--20\%.

    \item \textbf{Dark energy separates confusion}: Repulsive force between similar intents reduces misrouting by 30\%.

    \item \textbf{Brahim mathematics unifies}: Golden ratio constants provide consistent scaling across domains.
\end{enumerate}

\subsection{Future Work}

\begin{itemize}
    \item Extend to additional Cassini instruments (RADAR, MAG)
    \item Apply to other planetary databases (Mars, Europa)
    \item Implement real-time streaming extraction
    \item GPU acceleration for large-scale queries
\end{itemize}

%------------------------------------------------------------------------------
% REFERENCES
%------------------------------------------------------------------------------
\section*{References}
\addcontentsline{toc}{section}{References}

\begin{enumerate}[label={[\arabic*]}]
    \item NASA Planetary Data System. Cassini VIMS Data Archive. \url{https://pds.nasa.gov}

    \item SETI Institute. OPUS: Outer Planets Unified Search. \url{https://opus.pds-rings.seti.org}

    \item Lorenz, R.D. and Mitton, J. \textit{Titan Unveiled}. Princeton University Press, 2008.

    \item Livio, M. \textit{The Golden Ratio}. Broadway Books, 2002.

    \item Porco, C.C. et al. ``Imaging of Titan from the Cassini spacecraft.'' \textit{Nature} 434, 159--168 (2005).

    \item Brown, R.H. et al. ``The Cassini Visual and Infrared Mapping Spectrometer (VIMS) Investigation.'' \textit{Space Science Reviews} 115, 111--168 (2004).

    \item Flasar, F.M. et al. ``Exploring the Saturn System in the Thermal Infrared: The Composite Infrared Spectrometer.'' \textit{Space Science Reviews} 115, 169--297 (2004).
\end{enumerate}

%------------------------------------------------------------------------------
% APPENDIX
%------------------------------------------------------------------------------
\appendix
\section{OPUS ID Reference}

\begin{center}
\begin{tabular}{ll}
\toprule
\textbf{Prefix} & \textbf{Meaning}\\
\midrule
co- & Cassini Orbiter\\
vg- & Voyager\\
-iss- & Imaging Science Subsystem\\
-vims- & Visual/IR Mapping Spectrometer\\
-cirs- & Composite Infrared Spectrometer\\
-uvis- & Ultraviolet Imaging Spectrograph\\
\_ir & Infrared channel\\
\_vis & Visible channel\\
-n & Narrow-angle camera\\
-w & Wide-angle camera\\
\bottomrule
\end{tabular}
\end{center}

\section{Brahim Sequence Properties}

\begin{align}
    \mathcal{B} &= \{27, 42, 60, 75, 97, 121, 136, 154, 172, 187\}\\
    S &= \sum_{i=1}^{10} B_i = 214\\
    C &= 107 = S/2\\
    \frac{C}{S} &= \frac{1}{2} \quad \text{(critical line)}\\
    B_6 &= 121 \approx \text{Kelimutu longitude}
\end{align}

Mirror pairs: $M(x) = 214 - x$
\begin{itemize}[noitemsep]
    \item $M(27) = 187$ (in sequence)
    \item $M(42) = 172$ (in sequence)
    \item $M(60) = 154$ (in sequence)
    \item $M(75) = 139$ (NOT in sequence -- broken)
    \item $M(97) = 117$ (NOT in sequence -- broken)
\end{itemize}

\vfill
\begin{center}
\rule{0.5\textwidth}{0.4pt}\\[0.5cm]
{\small Document generated: January 2026}\\
{\small DOI: 10.5281/zenodo.18362052}\\
{\small License: Brahim Security License}
\end{center}

\end{document}
