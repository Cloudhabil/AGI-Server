% BSD Gap Closure Synthesis
% Generated by Dual-Model Orchestrator
% Date: 2026-01-05T03:29:22.977418

\documentclass{article}
\usepackage{amsmath, amssymb, amsthm}

\title{BSD Conjecture: Gap R1 and R2 Closure}
\author{GPIA Dual-Model Framework}
\date{\today}

\begin{document}
\maketitle

\section{Gap R1: Euler Systems Existence}
Progress: 0.0%
Model: qwen2-math:7b

\subsection{○ R1.1}
Rigor: 0.65


1. **State the claim precisely:**
   We need to prove that the wedge product in exterior algebras is compatible with the induced norm, specifically for any two elements \( \omega_1 \in \bigwedge^k V \) and \( \omega_2 \in \bigwedge^l V \), we need to show:
   \[
   \| \omega_1 \wedge \omega_2 \| = \| \omega_1 \| \cdot \| \omega_2 \|
   \]

2. **List proof steps with justification:**
   - **Step 1: Definition of Wedge Product:** Recall that the wedge product \( \omega_1 \wedge \omega_2 \) is defined in terms of the alternating sum over the tensor product:
     \[
     \omega_1 \wedge \omega_2 = \frac{1}{k! l!} \sum_{\sigma \in S_{k+l}} \text{sign}(\sigma) (\omega_1 \otimes \omega_2) \circ \sigma
     \]
   - **Step 2: Norm of the Wedge Product:** The norm of \( \omega_1 \wedge \omega_2 \) is given by:
     \[
     \| \omega_1 \wedge \omega_2 \| = \left( \sum_{i_1, \ldots, i_k, j_1, \ldots, j_l} | (\omega_1 \wedge \omega_2)(e_{i_1}, \ldots, e_{i_k}, e_{j_1}, \ldots, e_{j_l}) |^2 \right)


\section{Gap R2: Control Theorem Machinery}
Progress: 0.0%
Model: gpia-deepseek-r1:latest

\subsection{○ R2.1}
Rigor: 0.50




\section{Cross-Validation Summary}
Total validations: 4

\end{document}
