% BSD Gap 6 Closure Document
% Prize-Eligibility Packet Document 3
% STATUS: TEMPLATE - To be filled by crystallization cycles

\documentclass{article}
\usepackage{amsmath, amssymb, amsthm}
\usepackage{hyperref}

\newtheorem{theorem}{Theorem}
\newtheorem{lemma}[theorem]{Lemma}
\newtheorem{proposition}[theorem]{Proposition}
\newtheorem{corollary}[theorem]{Corollary}
\newtheorem{definition}[theorem]{Definition}
\newtheorem{conjecture}[theorem]{Conjecture}

\theoremstyle{remark}
\newtheorem{remark}[theorem]{Remark}

\title{BSD Conjecture: Gap 6 (Higher Rank) Closure}
\author{GPIA Meta-Analysis Research Framework}
\date{Draft - \today}

\begin{document}
\maketitle

\begin{abstract}
This document contains the core mathematical argument for resolving Gap 6
(Higher Rank case, $r \geq 2$) of the Birch and Swinnerton-Dyer Conjecture.
The argument proceeds via three attack vectors: Higher-Rank Euler Systems,
Derived Algebraic Geometry, and Infinity Folding. \textbf{STATUS: IN PROGRESS}
\end{abstract}

\tableofcontents

\section{Introduction}

\subsection{The Higher Rank Problem}

For elliptic curves $E/\mathbb{Q}$ with algebraic rank $r \geq 2$, the BSD conjecture
remains unproven. The fundamental difficulty is:

\begin{itemize}
    \item No ``Heegner-like'' points for higher ranks
    \item Kolyvagin's method fails for $r > 1$
    \item The $L$-function vanishing is of order $r$, creating a ``thick'' singularity
\end{itemize}

\subsection{Our Approach}

We pursue three parallel attack vectors, any one of which, if completed,
would resolve Gap 6.

%=============================================================================
\section{Vector 1: Higher-Rank Euler Systems}
%=============================================================================

\subsection{Statement}

\begin{conjecture}[Higher-Rank Euler System Existence]
For an elliptic curve $E/\mathbb{Q}$ with rank $r$, there exists an Euler system
$\mathbf{c} = (c_K)_K$ indexed by abelian extensions $K/\mathbb{Q}$ such that:
\begin{enumerate}
    \item $c_K \in \bigwedge^r H^1(K, T_p E)$
    \item The system satisfies a norm-compatibility relation
    \item A control theorem bounds the Selmer group
\end{enumerate}
\end{conjecture}

\subsection{Construction}

\textbf{[TO BE FILLED BY CRYSTALLIZATION CYCLES]}

% Placeholder for the actual construction
\begin{remark}
The construction should generalize Kato's Euler system (which works for $r=1$)
to systems indexed by $r$-tuples of Galois representations.
\end{remark}

\subsection{Control Theorem}

\begin{theorem}[Control Theorem - Template]
Let $\mathbf{c}$ be a higher-rank Euler system for $E$. Then:
\[
\text{length}_{\mathbb{Z}_p} \text{Sel}_{p^\infty}(E/\mathbb{Q}) \leq
\text{ord}_p(\text{some explicit quantity involving } \mathbf{c})
\]
\end{theorem}

\begin{proof}
\textbf{[PROOF TO BE COMPLETED]}
\end{proof}

%=============================================================================
\section{Vector 2: Derived Algebraic Geometry}
%=============================================================================

\subsection{The Derived Selmer Complex}

\begin{definition}[Derived Selmer Complex]
Define the Selmer complex as an object in the derived category:
\[
R\Gamma_f(E, \mathbb{Q}_p) \in D^b(\text{Mod}_{\mathbb{Z}_p})
\]
This complex has:
\begin{itemize}
    \item $H^0 = 0$
    \item $H^1 = \text{Sel}_{p^\infty}(E/\mathbb{Q})$
    \item $H^2 = $ dual Selmer
\end{itemize}
\end{definition}

\subsection{Virtual Dimension}

\begin{proposition}[Rank as Virtual Dimension]
The virtual dimension of the derived Selmer complex equals the rank:
\[
\text{vdim}(R\Gamma_f(E, \mathbb{Q}_p)) = \text{rank}_\mathbb{Z} E(\mathbb{Q})
\]
\end{proposition}

\begin{proof}
\textbf{[PROOF TO BE COMPLETED]}

The key steps are:
\begin{enumerate}
    \item Define virtual dimension via Euler characteristic
    \item Use Tate duality to relate $H^1$ and $H^2$
    \item Show $\chi = $ rank via Mordell-Weil
\end{enumerate}
\end{proof}

\subsection{Connection to $L$-function}

\begin{theorem}[Order = Virtual Dimension]
\[
\text{ord}_{s=1} L(E,s) = \text{vdim}(R\Gamma_f(E, \mathbb{Q}_p))
\]
\end{theorem}

\begin{proof}
\textbf{[PROOF TO BE COMPLETED]}

Requires the comparison morphism $\varphi$ from Cycle 46.
\end{proof}

%=============================================================================
\section{Vector 3: Infinity Folding}
%=============================================================================

\subsection{The Divergence Problem}

For $r \geq 2$, the height pairing series can diverge in classical terms.
The infinity folding technique re-expresses these as convergent $p$-adic series.

\subsection{Algorithm}

\begin{definition}[Infinity Folding Transform]
For a divergent real series $\sum a_n$ arising from height computations,
define the $p$-adic folded series:
\[
\mathcal{F}_p\left(\sum a_n\right) = \sum_{n=0}^\infty a_n \cdot \omega_p(n)
\]
where $\omega_p(n)$ is a $p$-adic weight function ensuring convergence.
\end{definition}

\subsection{Convergence}

\begin{theorem}[Infinity Folding Convergence]
The folded series converges in $\mathbb{Q}_p$ and equals the $p$-adic regulator.
\end{theorem}

\begin{proof}
\textbf{[PROOF TO BE COMPLETED]}
\end{proof}

\subsection{Computational Verification}

\begin{remark}
Scripts in \texttt{04\_reproducibility.md} verify this for curves with $r = 3, 4$.
\end{remark}

%=============================================================================
\section{Synthesis: Gap 6 Closure}
%=============================================================================

\subsection{Main Theorem}

\begin{theorem}[BSD for Higher Rank]
For any elliptic curve $E/\mathbb{Q}$ with rank $r \geq 2$:
\[
\text{rank}_\mathbb{Z} E(\mathbb{Q}) = \text{ord}_{s=1} L(E,s)
\]
\end{theorem}

\begin{proof}
\textbf{[SYNTHESIS PROOF TO BE COMPLETED]}

The proof combines:
\begin{enumerate}
    \item Vector 1 (Euler Systems) OR Vector 2 (Derived AG)
    \item Vector 3 (Infinity Folding) for explicit computation
    \item The Comparison Morphism $\varphi$ as the bridge
\end{enumerate}
\end{proof}

%=============================================================================
\section{Sha Finiteness}
%=============================================================================

\begin{theorem}[Sha Finiteness for All Ranks]
$\#\text{Ш}(E/\mathbb{Q}) < \infty$ for all $E/\mathbb{Q}$.
\end{theorem}

\begin{proof}
\textbf{[PROOF TO BE COMPLETED]}

This follows from Gap 6 closure via the control theorem.
\end{proof}

%=============================================================================
\section{Conclusion}
%=============================================================================

This document will be complete when at least one vector achieves full proof status
and the synthesis theorem is established. Current progress is tracked in the
crystallization cycles.

\appendix

\section{Notation Index}
\begin{itemize}
    \item $E$: Elliptic curve over $\mathbb{Q}$
    \item $r$: Algebraic rank $= \text{rank}_\mathbb{Z} E(\mathbb{Q})$
    \item $L(E,s)$: Hasse-Weil $L$-function
    \item $T_p E$: $p$-adic Tate module
    \item $\text{Sel}_{p^\infty}$: $p^\infty$-Selmer group
    \item $R\Gamma_f$: Derived Selmer complex
    \item $\varphi$: Comparison morphism
\end{itemize}

\end{document}
