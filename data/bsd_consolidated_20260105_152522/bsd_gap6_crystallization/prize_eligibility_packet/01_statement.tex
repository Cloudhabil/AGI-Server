% BSD Conjecture - Precise Statement
% Prize-Eligibility Packet Document 1
% Generated by GPIA Meta-Analysis Research Framework

\documentclass{article}
\usepackage{amsmath, amssymb, amsthm}
\usepackage{hyperref}

\newtheorem{conjecture}{Conjecture}
\newtheorem{definition}{Definition}

\title{Birch and Swinnerton-Dyer Conjecture: Precise Statement}
\author{GPIA Meta-Analysis Research Framework}
\date{\today}

\begin{document}
\maketitle

\section{Statement of the BSD Conjecture}

Let $E$ be an elliptic curve defined over $\mathbb{Q}$.

\begin{conjecture}[BSD Weak Form]
The algebraic rank of $E(\mathbb{Q})$ equals the analytic rank:
\[
\text{rank}_{\mathbb{Z}} E(\mathbb{Q}) = \text{ord}_{s=1} L(E,s)
\]
where $L(E,s)$ is the Hasse-Weil $L$-function of $E$.
\end{conjecture}

\begin{conjecture}[BSD Strong Form]
The leading coefficient of $L(E,s)$ at $s=1$ satisfies:
\[
\lim_{s \to 1} \frac{L(E,s)}{(s-1)^r} = \frac{\#\text{Ш}(E/\mathbb{Q}) \cdot \Omega_E \cdot \text{Reg}(E) \cdot \prod_{p} c_p}{(\#E(\mathbb{Q})_{\text{tors}})^2}
\]
where $r = \text{rank}_{\mathbb{Z}} E(\mathbb{Q})$.
\end{conjecture}

\section{Definitions and Normalizations}

\subsection{Curve Class}
We consider elliptic curves $E/\mathbb{Q}$ given in minimal Weierstrass form:
\[
y^2 + a_1 xy + a_3 y = x^3 + a_2 x^2 + a_4 x + a_6, \quad a_i \in \mathbb{Z}
\]
with minimal discriminant $\Delta_E$ and conductor $N_E$.

\subsection{$L$-Function Normalization}
The $L$-function is defined as the Euler product:
\[
L(E,s) = \prod_{p \nmid N_E} \frac{1}{1 - a_p p^{-s} + p^{1-2s}} \cdot \prod_{p | N_E} \frac{1}{1 - a_p p^{-s}}
\]
where $a_p = p + 1 - \#E(\mathbb{F}_p)$ for good reduction.

By modularity (Wiles et al.), $L(E,s)$ extends to an entire function on $\mathbb{C}$ and satisfies the functional equation:
\[
\Lambda(E,s) = w_E \Lambda(E, 2-s)
\]
where $\Lambda(E,s) = N_E^{s/2} (2\pi)^{-s} \Gamma(s) L(E,s)$ and $w_E = \pm 1$ is the root number.

\subsection{Real Period}
\[
\Omega_E = \int_{E(\mathbb{R})} |\omega|
\]
where $\omega = \frac{dx}{2y + a_1 x + a_3}$ is the N\'eron differential.

\subsection{Regulator}
The regulator is defined as:
\[
\text{Reg}(E) = \det \left( \langle P_i, P_j \rangle_{\text{NT}} \right)_{1 \leq i,j \leq r}
\]
where $\{P_1, \ldots, P_r\}$ is any basis of $E(\mathbb{Q})/E(\mathbb{Q})_{\text{tors}}$ and $\langle \cdot, \cdot \rangle_{\text{NT}}$ is the N\'eron-Tate height pairing.

\textbf{Convention:} $\text{Reg}(E) = 1$ if $r = 0$.

\subsection{Tamagawa Numbers}
For each prime $p$ of bad reduction:
\[
c_p = [E(\mathbb{Q}_p) : E^0(\mathbb{Q}_p)]
\]
where $E^0(\mathbb{Q}_p)$ is the subgroup of points with non-singular reduction.

\subsection{Tate-Shafarevich Group}
\[
\text{Ш}(E/\mathbb{Q}) = \ker \left( H^1(\mathbb{Q}, E) \to \prod_v H^1(\mathbb{Q}_v, E) \right)
\]
where the product runs over all places of $\mathbb{Q}$.

\textbf{Assumption:} We assume $\text{Ш}(E/\mathbb{Q})$ is finite. This is conjectured but not proven in general.

\section{What We Claim to Prove}

\textbf{Target:} Both BSD Weak Form and Strong Form for all elliptic curves $E/\mathbb{Q}$.

\textbf{Gap 6 Status:} The case $r \geq 2$ (higher rank) remains the critical obstacle. Rank 0 and 1 cases follow from Gross-Zagier and Kolyvagin.

\section{Key Quantities Summary}

\begin{center}
\begin{tabular}{|c|l|l|}
\hline
\textbf{Symbol} & \textbf{Name} & \textbf{Type} \\
\hline
$r$ & Algebraic rank & $\mathbb{Z}_{\geq 0}$ \\
$\text{ord}_{s=1} L(E,s)$ & Analytic rank & $\mathbb{Z}_{\geq 0}$ \\
$\Omega_E$ & Real period & $\mathbb{R}_{>0}$ \\
$\text{Reg}(E)$ & Regulator & $\mathbb{R}_{>0}$ \\
$c_p$ & Tamagawa number at $p$ & $\mathbb{Z}_{>0}$ \\
$\#\text{Ш}$ & Sha order & $\mathbb{Z}_{>0}$ (conjectured finite) \\
$\#E_{\text{tors}}$ & Torsion order & $\mathbb{Z}_{>0}$ \\
\hline
\end{tabular}
\end{center}

\end{document}
